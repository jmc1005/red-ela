\apendice{Especificación de Requisitos}

\section{Introducción}

En este punto vamos a definir los requisitos que va a tener nuestra primera versión del proyecto. Para ello vamos a seguir la \textbf{especificación de requisitos de sofware (ERS)}~\cite{wiki:ieee830-1998}, donde se recogen una serie de buenas prácticas y son las siguientes:
\begin{itemize}
\tightlist 
\item \textbf{Completa}. Todos los requerimientos deben estar contemplados en ella y deben estar definidas todas las referencias .
\item \textbf{Consistente}. Debe tener coherencia con los requerimientos y además, con otras especificaciones.
\item \textbf{Inequívoca}. Claridad a la hora de redactar el documento para que no haya malas interpretaciones.
\item \textbf{Correcta}. Se deben cumplir con los requisitos recogidos en la especificación.
\item \textbf{Trazable}. Hace referencia a la posibilidad de comprobar la historia, aplicación o ubicación de un item mediante su identificación almacenada y documentada.
\item \textbf{Priorizable}. Posibilidad de organizar los items según la relevancia y separándolos en esenciales, condicionales y opcionales.
\item \textbf{Modificable}. Dar la posibilidad de poder ser modificable.
\item \textbf{Verificable}. Tener la posibilidad de realizar las pruebas de manera finita y sin costo.
\item \textbf{Clara}. Debe ser entendible.
\end{itemize}

\section{Objetivos generales}
Nuestro proyecto tendrá como objetivos principales:
\begin{itemize}
\tightlist 
\item Crear una aplicación multiplataforma en Android y Web.
\item Gestionar las citas a través de los gestores de casos y asignárselas a pacientes con ELA.
\item Visualizar las citas que tengan en el día.
\item Visualizar las citas que tengan en un calendario.
\item Gestionar aquella información necesaria para los pacientes, como tratamientos, hospitales.
\end{itemize}
\section{Catálogo de requisitos}
Vamos a hacer en el catálogo de requisitos una mención de los requisitos específicos tras el análisis de los objetivos generales.

\subsection{Requisitos Funcionales (RF)}
\subsubsection{RF-1 Gestión Gestor Casos}
\begin{itemize}
\tightlist 
\item \textbf{RF-1.1 Invitar gestor casos}. El gestor de casos puede enviar una invitación para que un gestor de casos se registre en la aplicación.
\item \textbf{RF-1.2 Invitar paciente}. El gestor de casos puede enviar una invitación para que el paciente se registre en la aplicación.
\item \textbf{RF-1.3 Eliminar paciente}. El paciente podrá ser eliminado por parte del gestor de casos.
\item \textbf{RF-1.4 Listar pacientes}. El gestor de casos podrá ver un listado de pacientes asignados a él.
\item \textbf{RF-1.5 Ver paciente}. El gestor de casos podrá ver los datos del paciente seleccionado.
\item \textbf{RF-1.6 Editar gestor casos}. Podrá actualizar sus datos.
\item \textbf{RF-1.7 Ver citas}. Ver citas.
\item \textbf{RF-1.8 Crear cita}. Podrá crear cita.
\item \textbf{RF-1.9 Editar cita}. Podrá crear cita.
\item \textbf{RF-1.10 Cancelar cita}. Podrá cancelar cita.
\end{itemize}

\subsubsection{RF-2 Gestión Paciente}
\begin{itemize}
\tightlist 
\item \textbf{RF-2.1 Editar paciente}. El paciente será el encargado de gestionar sus datos.
\item \textbf{RF-2.2 Invitar cuidador principal}. El paciente puede enviar una invitación para que el cuidador se registre en la aplicación.
\item \textbf{RF-2.3 Ver usuarios asociados}. El paciente podrá ver la información del cuidador y el gestor de casos asociados.
\item \textbf{RF-2.4 Cancelar cita}. Podrá cancelar cita.
\end{itemize}

\subsubsection{RF-3 Gestión Cuidador Principal}
\begin{itemize}
\tightlist 
\item \textbf{RF-3.1 Editar cuidador}. El paciente será el encargado de gestionar sus datos.
\item \textbf{RF-3.2 Invitar cuidador principal}. El paciente puede enviar una invitación para que el cuidador se registre en la aplicación.
\item \textbf{RF-3.3 Ver pacientes asociados}. El cuidador podrá ver la información del paciente.
\item \textbf{RF-3.4 Cancelar cita}. Podrá cancelar cita.
\end{itemize}

\subsubsection{RF-4 Gestión Citas}
\begin{itemize}
\tightlist 
\item \textbf{RF-4.1 Crear cita}. Crear una cita nueva.
\item \textbf{RF-4.2 Cancelar cita}. Eliminar cita.
\item \textbf{RF-4.3 Listar citas}. Ver las citas creadas.
\item \textbf{RF-4.3 Editar cita}. Editar la cita.
\end{itemize}

\subsubsection{RF-5 Gestión Admin}
\begin{itemize}
\tightlist 
\item \textbf{RF-5.1 Invitar admin}. Envía invitación para que un usuario pueda registrarse como admin.
\item \textbf{RF-5.2 Invitar gestor casos}. Envía invitación para que un usuario pueda registrarse como gestor de casos.
\item \textbf{RF-5.3 Listar admin}. Puede ver los administradores de la aplicación.
\item \textbf{RF-5.4 Listar gestores de casos}. Puede ver los gestores de casos de la aplicación.
\item \textbf{RF-5.5 Listar pacientes}. Puede ver los pacientes de la aplicación.
\item \textbf{RF-5.6 Listar cuidadores}. Puede ver los cuidadores de la aplicación.
\item \textbf{RF-5.7 Eliminar usuario}. Puede eliminar usuario.
\item \textbf{RF-5.8 Ver usuario}. Ver usuario.
\item \textbf{RF-5.9 Listar hospitales}. Ver las hospitales.
\item \textbf{RF-5.10 Crear hospital}. Crear hospital.
\item \textbf{RF-5.11 Editar hospital}. Editar hospital.
\item \textbf{RF-5.12 Eliminar hospital}. Eliminar hospital.
\item \textbf{RF-5.13 Listar tratamientos}. Ver las tratamientos.
\item \textbf{RF-5.14 Crear tratamiento}. Crear tratamiento.
\item \textbf{RF-5.16 Editar tratamiento}. Editar tratamiento.
\item \textbf{RF-5.17 Eliminar tratamiento}. Eliminar tratamiento.
\item \textbf{RF-5.18 Editar admin}. Editar admin.
\end{itemize}

\section{Especificación de requisitos}
\subsection{Diagramas de flujo}
\imagen{admin-flujo}{Diagrama flujo admin}
\imagen{gestor-casos-flujo}{Diagrama flujo gestor casos}
\imagen{paciente-cuidador-flujo}{Diagrama flujo paciente/cuidador}
\subsection{Casos de Uso}
\begin{table}[p]
	\centering
	\begin{tabularx}{\linewidth}{ p{0.21\columnwidth} p{0.71\columnwidth} }
		\toprule
		\textbf{CU-1}    & \textbf{Gestión usuarios}\\
		\toprule
		\textbf{Versión}              & 1.0    \\
		\textbf{Autor}                & {\nombre}\\
		\textbf{Requisitos asociados} & RF-1, RF-2, RF-3, RF-5\\
		\textbf{Descripción}          & Permite gestionar los gestores de casos \\
		\textbf{Precondición}         & Firebase está disponible \\
		\textbf{Acciones}             &
		\begin{enumerate}
			\def\labelenumi{\arabic{enumi}.}
			\tightlist
			\item Accede a la aplicación
			\item Ver información personal
			\item Todos menos el admin ver las citas
			\item Todos menos el admin ver usuarios asociados
			\item Si es gestor o admin ver botón invitar a usuario
			\item Solo admin ver usuarios (administradores, gestores casos, pacientes y cuidadores principales)
			\item Solo admin ver tratamientos
			\item Solo admin ver hospitales
		\end{enumerate}\\
		\textbf{Postcondición}        & Los pacientes y citas coinciden con las que hay en la base de datos \\
		\textbf{Excepciones}          & Error al obtener los pacientes o citas \\
		\textbf{Importancia}          & Alta \\
		\bottomrule
	\end{tabularx}
	\caption{CU-1 Gestión Gestores de casos.}
\end{table}

\begin{table}[p]
	\centering
	\begin{tabularx}{\linewidth}{ p{0.21\columnwidth} p{0.71\columnwidth} }
		\toprule
		\textbf{CU-2}    & \textbf{Enviar invitación}\\
		\toprule
		\textbf{Versión}              & 1.0    \\
		\textbf{Autor}                & {\nombre} \\
		\textbf{Requisitos asociados} & RF-1.1, RF-1.2, RF-2.2, RF-3.2, RF-5.1, RF-5.2 \\
		\textbf{Descripción}          & Permite invitar a un usuario \\
		\textbf{Precondición}         & Firebase está disponible \\
		\textbf{Acciones}             &
		\begin{enumerate}
			\def\labelenumi{\arabic{enumi}.}
			\tightlist
			\item Accede a la aplicación
			\item Ir a la pestaña de usuarios
			\item Pulsar el botón (+) de añadir
			\item Sale un pop up para rellenar los datos
			\item Enviar invitación
		\end{enumerate}\\
		\textbf{Postcondición}        & La invitación es recibida por el usuario final \\
		\textbf{Excepciones}          & Error al enviar invitación \\
		\textbf{Importancia}          & Alta \\
		\bottomrule
	\end{tabularx}
	\caption{CU-2 Enviar invitación.}
\end{table}

\begin{table}[p]
	\centering
	\begin{tabularx}{\linewidth}{ p{0.21\columnwidth} p{0.71\columnwidth} }
		\toprule
		\textbf{CU-3}    & \textbf{Eliminar usuario}\\
		\toprule
		\textbf{Versión}              & 1.0    \\
		\textbf{Autor}                & {\nombre} \\
		\textbf{Requisitos asociados} & RF-1.3, RF-5.7 \\
		\textbf{Descripción}          & Permite eliminar un usuario \\
		\textbf{Precondición}         & Firebase está disponible \\
		\textbf{Acciones}             &
		\begin{enumerate}
			\def\labelenumi{\arabic{enumi}.}
			\tightlist
			\item Accede a la aplicación
			\item Ir a la pestaña de usuarios
			\item Pulsar papelera para borrar
		\end{enumerate}\\
		\textbf{Postcondición}        & El usuario se borra \\
		\textbf{Excepciones}          & Error al borrar \\
		\textbf{Importancia}          & Alta \\
		\bottomrule
	\end{tabularx}
	\caption{CU-3 Eliminar usuario.}
\end{table}

\begin{table}[p]
	\centering
	\begin{tabularx}{\linewidth}{ p{0.21\columnwidth} p{0.71\columnwidth} }
		\toprule
		\textbf{CU-4}    & \textbf{Listar usuarios}\\
		\toprule
		\textbf{Versión}              & 1.0    \\
		\textbf{Autor}                & {\nombre} \\
		\textbf{Requisitos asociados} & RF-1.4, RF-5.3, RF-5.4, RF-5.5, RF-5.6 \\
		\textbf{Descripción}          & Permite listar usuarios \\
		\textbf{Precondición}         & Firebase está disponible \\
		\textbf{Acciones}             &
		\begin{enumerate}
			\def\labelenumi{\arabic{enumi}.}
			\tightlist
			\item Accede a la aplicación
			\item Ir a la pestaña de usuarios
		\end{enumerate}\\
		\textbf{Postcondición}        & Se muestran los usuarios \\
		\textbf{Excepciones}          & Error al obtener listado \\
		\textbf{Importancia}          & Alta \\
		\bottomrule
	\end{tabularx}
	\caption{CU-4 Listar usuarios.}
\end{table}

\begin{table}[p]
	\centering
	\begin{tabularx}{\linewidth}{ p{0.21\columnwidth} p{0.71\columnwidth} }
		\toprule
		\textbf{CU-5}    & \textbf{Ver usuario}\\
		\toprule
		\textbf{Versión}              & 1.0    \\
		\textbf{Autor}                & {\nombre} \\
		\textbf{Requisitos asociados} & RF-1.5, RF-2.3, RF-3.3, RF-5.8 \\
		\textbf{Descripción}          & Permite ver un usuario \\
		\textbf{Precondición}         & Firebase está disponible \\
		\textbf{Acciones}             &
		\begin{enumerate}
			\def\labelenumi{\arabic{enumi}.}
			\tightlist
			\item Accede a la aplicación
			\item Ir a la pestaña de usuarios
			\item Pulsar en usuario para ver la información
		\end{enumerate}\\
		\textbf{Postcondición}        & El usuario se muestra con la información correcta \\
		\textbf{Excepciones}          & Error al obtener los datos\\
		\textbf{Importancia}          & Alta \\
		\bottomrule
	\end{tabularx}
	\caption{CU-5 Ver usuario.}
\end{table}

\begin{table}[p]
	\centering
	\begin{tabularx}{\linewidth}{ p{0.21\columnwidth} p{0.71\columnwidth} }
		\toprule
		\textbf{CU-6}    & \textbf{Editar usuario}\\
		\toprule
		\textbf{Versión}              & 1.0    \\
		\textbf{Autor}                & {\nombre} \\
		\textbf{Requisitos asociados} & RF-1.6, RF-2.1, RF-3.1, RF-5.8, RF-5.18 \\
		\textbf{Descripción}          & Permite editar un usuario \\
		\textbf{Precondición}         & Firebase está disponible \\
		\textbf{Acciones}             &
		\begin{enumerate}
			\def\labelenumi{\arabic{enumi}.}
			\tightlist
			\item Accede a la aplicación
			\item Pulsar en el icono usuario
			\item Editar la información
		\end{enumerate}\\
		\textbf{Postcondición}        & El usuario se borra \\
		\textbf{Excepciones}          & Error al borrar \\
		\textbf{Importancia}          & Alta \\
		\bottomrule
	\end{tabularx}
	\caption{CU-6 Editar usuario.}
\end{table}

\begin{table}[p]
	\centering
	\begin{tabularx}{\linewidth}{ p{0.21\columnwidth} p{0.71\columnwidth} }
		\toprule
		\textbf{CU-7}    & \textbf{Gestión citas}\\
		\toprule
		\textbf{Versión}              & 1.0    \\
		\textbf{Autor}                & {\nombre} \\
		\textbf{Requisitos asociados} & RF-1.8, RF-1.9, RF-1.10, RF-2.4, RF-3.4, RF-4 \\
		\textbf{Descripción}          & Permite ver las citas\\
		\textbf{Precondición}         & Firebase está disponible \\
		\textbf{Acciones}             &
		\begin{enumerate}
			\def\labelenumi{\arabic{enumi}.}
			\tightlist
			\item Accede a la aplicación
			\item Ver citas asociadas del día en la home
			\item Ver todas las citas en el calendario
		\end{enumerate}\\
		\textbf{Postcondición}        & Se muestran las citas \\
		\textbf{Excepciones}          & Error al obtener los datos \\
		\textbf{Importancia}          & Alta \\
		\bottomrule
	\end{tabularx}
	\caption{CU-7 Gestión citas.}
\end{table}

\begin{table}[p]
	\centering
	\begin{tabularx}{\linewidth}{ p{0.21\columnwidth} p{0.71\columnwidth} }
		\toprule
		\textbf{CU-8}    & \textbf{Crear cita}\\
		\toprule
		\textbf{Versión}              & 1.0    \\
		\textbf{Autor}                & {\nombre} \\
		\textbf{Requisitos asociados} & RF-1.8, RF-4.1 \\
		\textbf{Descripción}          & Permite crear una cita\\
		\textbf{Precondición}         & Firebase está disponible \\
		\textbf{Acciones}             &
		\begin{enumerate}
			\def\labelenumi{\arabic{enumi}.}
			\tightlist
			\item Accede a la aplicación
			\item Ir al calendario
			\item Pulsar botón (+) para crear cita
			\item Introducir los datos
			\item Aceptar
		\end{enumerate}\\
		\textbf{Postcondición}        & Se crea la cita \\
		\textbf{Excepciones}          & Error al obtener los datos \\
		\textbf{Importancia}          & Alta \\
		\bottomrule
	\end{tabularx}
	\caption{CU-8 Crear cita.}
\end{table}


\begin{table}[p]
	\centering
	\begin{tabularx}{\linewidth}{ p{0.21\columnwidth} p{0.71\columnwidth} }
		\toprule
		\textbf{CU-9}    & \textbf{Editar cita}\\
		\toprule
		\textbf{Versión}              & 1.0    \\
		\textbf{Autor}                & {\nombre} \\
		\textbf{Requisitos asociados} & RF-1.9 RF-4.4 \\
		\textbf{Descripción}          & Permite editar una cita\\
		\textbf{Precondición}         & Firebase está disponible \\
		\textbf{Acciones}             &
		\begin{enumerate}
			\def\labelenumi{\arabic{enumi}.}
			\tightlist
			\item Accede a la aplicación
			\item Ir al calendario
			\item Pulsar sobre una cita
			\item Cambiar los datos
			\item Aceptar
		\end{enumerate}\\
		\textbf{Postcondición}        & Se modifica la cita \\
		\textbf{Excepciones}          & Error al obtener los datos \\
		\textbf{Importancia}          & Alta \\
		\bottomrule
	\end{tabularx}
	\caption{CU-9 Editar cita.}
\end{table}

\begin{table}[p]
	\centering
	\begin{tabularx}{\linewidth}{ p{0.21\columnwidth} p{0.71\columnwidth} }
		\toprule
		\textbf{CU-10}    & \textbf{Cancelar cita}\\
		\toprule
		\textbf{Versión}              & 1.0    \\
		\textbf{Autor}                & {\nombre} \\
		\textbf{Requisitos asociados} & RF-1.10, RF-2.4, RF-3.4, RF-4.2 \\
		\textbf{Descripción}          & Permite cancelar una cita\\
		\textbf{Precondición}         & Firebase está disponible \\
		\textbf{Acciones}             &
		\begin{enumerate}
			\def\labelenumi{\arabic{enumi}.}
			\tightlist
			\item Accede a la aplicación
			\item Ir al calendario
			\item Pulsar sobre una cita
			\item Pulsar sobre cancelar cita
		\end{enumerate}\\
		\textbf{Postcondición}        & Se cancela la cita \\
		\textbf{Excepciones}          & Error al obtener los datos \\
		\textbf{Importancia}          & Alta \\
		\bottomrule
	\end{tabularx}
	\caption{CU-10 Cancelar cita.}
\end{table}


\begin{table}[p]
	\centering
	\begin{tabularx}{\linewidth}{ p{0.21\columnwidth} p{0.71\columnwidth} }
		\toprule
		\textbf{CU-11}    & \textbf{Listar hospitales}\\
		\toprule
		\textbf{Versión}              & 1.0    \\
		\textbf{Autor}                & {\nombre} \\
		\textbf{Requisitos asociados} & RF-5.9 \\
		\textbf{Descripción}          & Permite ver los hospitales \\
		\textbf{Precondición}         & Firebase está disponible \\
		\textbf{Acciones}             &
		\begin{enumerate}
			\def\labelenumi{\arabic{enumi}.}
			\tightlist
			\item Accede a la aplicación como admin
			\item Pulsar sobre hospitales
		\end{enumerate}\\
		\textbf{Postcondición}        & Se obtienen los hospitales\\
		\textbf{Excepciones}          & Error al obtener los datos \\
		\textbf{Importancia}          & Alta \\
		\bottomrule
	\end{tabularx}
	\caption{CU-11 Listar hospitales.}
\end{table}

\begin{table}[p]
	\centering
	\begin{tabularx}{\linewidth}{ p{0.21\columnwidth} p{0.71\columnwidth} }
		\toprule
		\textbf{CU-12}    & \textbf{Listar tratamientos}\\
		\toprule
		\textbf{Versión}              & 1.0    \\
		\textbf{Autor}                & {\nombre} \\
		\textbf{Requisitos asociados} & RF-5.13 \\
		\textbf{Descripción}          & Permite ver los tratamientos \\
		\textbf{Precondición}         & Firebase está disponible \\
		\textbf{Acciones}             &
		\begin{enumerate}
			\def\labelenumi{\arabic{enumi}.}
			\tightlist
			\item Accede a la aplicación como admin
			\item Pulsar sobre tratamientos
		\end{enumerate}\\
		\textbf{Postcondición}        & Se obtienen los tratamientos\\
		\textbf{Excepciones}          & Error al obtener los datos \\
		\textbf{Importancia}          & Alta \\
		\bottomrule
	\end{tabularx}
	\caption{CU-12 Listar tratamientos.}
\end{table}


\begin{table}[p]
	\centering
	\begin{tabularx}{\linewidth}{ p{0.21\columnwidth} p{0.71\columnwidth} }
		\toprule
		\textbf{CU-13}    & \textbf{Crear hospital}\\
		\toprule
		\textbf{Versión}              & 1.0    \\
		\textbf{Autor}                & {\nombre} \\
		\textbf{Requisitos asociados} & RF-5.10 \\
		\textbf{Descripción}          & Permite crear un hospital \\
		\textbf{Precondición}         & Firebase está disponible \\
		\textbf{Acciones}             &
		\begin{enumerate}
			\def\labelenumi{\arabic{enumi}.}
			\tightlist
			\item Accede a la aplicación como admin
			\item Pulsar sobre hospitales
			\item Pulsar sobre el botón (+)
			\item Introducir los datos
			\item Aceptar
		\end{enumerate}\\
		\textbf{Postcondición}        & Se guarda el hospital\\
		\textbf{Excepciones}          & Error al obtener los datos \\
		\textbf{Importancia}          & Alta \\
		\bottomrule
	\end{tabularx}
	\caption{CU-13 Crear hospital.}
\end{table}

\begin{table}[p]
	\centering
	\begin{tabularx}{\linewidth}{ p{0.21\columnwidth} p{0.71\columnwidth} }
		\toprule
		\textbf{CU-14}    & \textbf{Crear tratamiento}\\
		\toprule
		\textbf{Versión}              & 1.0    \\
		\textbf{Autor}                & {\nombre} \\
		\textbf{Requisitos asociados} & RF-5.14 \\
		\textbf{Descripción}          & Permite crear un tratamiento \\
		\textbf{Precondición}         & Firebase está disponible \\
		\textbf{Acciones}             &
		\begin{enumerate}
			\def\labelenumi{\arabic{enumi}.}
			\tightlist
			\item Accede a la aplicación como admin
			\item Pulsar sobre tratamientos
			\item Pulsar sobre el botón (+)
			\item Introducir los datos
			\item Aceptar
		\end{enumerate}\\
		\textbf{Postcondición}        & Se guarda el tratamiento\\
		\textbf{Excepciones}          & Error al obtener los datos \\
		\textbf{Importancia}          & Alta \\
		\bottomrule
	\end{tabularx}
	\caption{CU-14 Crear tratamiento.}
\end{table}

\begin{table}[p]
	\centering
	\begin{tabularx}{\linewidth}{ p{0.21\columnwidth} p{0.71\columnwidth} }
		\toprule
		\textbf{CU-15}    & \textbf{Editar hospital}\\
		\toprule
		\textbf{Versión}              & 1.0    \\
		\textbf{Autor}                & {\nombre} \\
		\textbf{Requisitos asociados} & RF-5.11 \\
		\textbf{Descripción}          & Permite crear un hospital \\
		\textbf{Precondición}         & Firebase está disponible \\
		\textbf{Acciones}             &
		\begin{enumerate}
			\def\labelenumi{\arabic{enumi}.}
			\tightlist
			\item Accede a la aplicación como admin
			\item Pulsar sobre hospitales
			\item Pulsar sobre un hospital
			\item Introducir los datos
			\item Aceptar
		\end{enumerate}\\
		\textbf{Postcondición}        & Se guarda el hospital\\
		\textbf{Excepciones}          & Error al obtener los datos \\
		\textbf{Importancia}          & Alta \\
		\bottomrule
	\end{tabularx}
	\caption{CU-15 Editar hospital.}
\end{table}

\begin{table}[p]
	\centering
	\begin{tabularx}{\linewidth}{ p{0.21\columnwidth} p{0.71\columnwidth} }
		\toprule
		\textbf{CU-16}    & \textbf{Editar tratamiento}\\
		\toprule
		\textbf{Versión}              & 1.0    \\
		\textbf{Autor}                & {\nombre} \\
		\textbf{Requisitos asociados} & RF-5.15 \\
		\textbf{Descripción}          & Permite editar un tratamiento \\
		\textbf{Precondición}         & Firebase está disponible \\
		\textbf{Acciones}             &
		\begin{enumerate}
			\def\labelenumi{\arabic{enumi}.}
			\tightlist
			\item Accede a la aplicación como admin
			\item Pulsar sobre tratamientos
			\item Pulsar sobre un tratamiento
			\item Introducir los datos
			\item Aceptar
		\end{enumerate}\\
		\textbf{Postcondición}        & Se guarda el tratamiento\\
		\textbf{Excepciones}          & Error al obtener los datos \\
		\textbf{Importancia}          & Alta \\
		\bottomrule
	\end{tabularx}
	\caption{CU-16 Editar tratamiento.}
\end{table}

\begin{table}[p]
	\centering
	\begin{tabularx}{\linewidth}{ p{0.21\columnwidth} p{0.71\columnwidth} }
		\toprule
		\textbf{CU-17}    & \textbf{Eliminar hospital}\\
		\toprule
		\textbf{Versión}              & 1.0    \\
		\textbf{Autor}                & {\nombre} \\
		\textbf{Requisitos asociados} & RF-5.12 \\
		\textbf{Descripción}          & Permite eliminar un hospital \\
		\textbf{Precondición}         & Firebase está disponible \\
		\textbf{Acciones}             &
		\begin{enumerate}
			\def\labelenumi{\arabic{enumi}.}
			\tightlist
			\item Accede a la aplicación como admin
			\item Pulsar sobre hospitales
			\item Pulsar sobre un hospital
			\item pulsar sobre la papelera
			\item Aceptar
		\end{enumerate}\\
		\textbf{Postcondición}        & Se elimina el hospital\\
		\textbf{Excepciones}          & Error al obtener los datos \\
		\textbf{Importancia}          & Alta \\
		\bottomrule
	\end{tabularx}
	\caption{CU-17 Eliminar hospital.}
\end{table}

\begin{table}[p]
	\centering
	\begin{tabularx}{\linewidth}{ p{0.21\columnwidth} p{0.71\columnwidth} }
		\toprule
		\textbf{CU-18}    & \textbf{Eliminar tratamiento}\\
		\toprule
		\textbf{Versión}              & 1.0    \\
		\textbf{Autor}                & {\nombre} \\
		\textbf{Requisitos asociados} & RF-5.17 \\
		\textbf{Descripción}          & Permite eliminar un tratamiento \\
		\textbf{Precondición}         & Firebase está disponible \\
		\textbf{Acciones}             &
		\begin{enumerate}
			\def\labelenumi{\arabic{enumi}.}
			\tightlist
			\item Accede a la aplicación como admin
			\item Pulsar sobre tratamientos
			\item Pulsar sobre la papelera
			\item Aceptar
		\end{enumerate}\\
		\textbf{Postcondición}        & Se elimina el tratamiento\\
		\textbf{Excepciones}          & Error al obtener los datos \\
		\textbf{Importancia}          & Alta \\
		\bottomrule
	\end{tabularx}
	\caption{CU-18 Eliminar tratamiento.}
\end{table}