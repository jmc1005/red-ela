\capitulo{1}{Introducción}

Incialmente el proyecto nacía para dar solución y mejorar la calidad de vida de personas que tengan \textbf{Esclerosis Múltiple}~\cite{wiki:em,medlineplus:em} (\textbf{EM}). Para ello se iba a desarrollar el Frontend inicialmente para dispositivo móvil Android. 

Resumiendo, podemos decir que la EM es una enfermedad neurológica crónica cuya naturaleza es inflamatoria y autoinmunitaria. Se caracteriza por la afectar el cerebro y la médula espinal y hace más lentos o bloquea la conexión entre el cerebro y el cuerpo.
Algunos síntomas de EM pueden ser:
\begin{itemize}
\item Alteraciones en la visión.
\item Debilidad a nivel muscular.
\item Problemas con la coordinación y el equilibrio.
\item Apacición de lesiones cerebrales detectadas en la resonancia magnética.
\end{itemize}
Podemos clasificar esta enfermedad entre:
\begin{itemize}
\item \textbf{Recurrente-remitente (EMRR)} caracterizada por aparición periódica de ataques o brotes.
\item \textbf{Secundaria progresiva (EMSP)} se caracteriza por el empeoramiento gradual y progresiva de la discapacidad.
\item \textbf{Primaria progresiva (EMPP)} este tipo comienza de manera inofensiva y empeora con el tiempo.
\item \textbf{Progresiva-recurrente (EMPR)} este es el más agresivo ya que las apariciones de los síntomas son severas y periódicas.
\end{itemize}

Ante los problemas de diseño para poder desarrollar la aplicación y con la aprobación del tutor, {\nomtutor}, nos ponemos en contacto con al asociación de ELA de Castilla y León  y realizamos una primera toma de contacto con ellos para saber que objetivos iniciales les parecería interesante para crear una aplicación multiplataforma que pudiera ayudar a pacientes o personas afectadas con la enfermedad. Al mismo tiempo, investigamos un poco de que trata la enfermedad y podemos afirmar que, la ELA que es una enfermedad degenerativa que daña el sistema nervioso periférico y central. Va degenerando de manera progresiva las células motoras del cerebro y la médula espinal.

Tras analizar los objetivos marcados por la asociación y ver que la aplicación contará inicialmente con tres actores que serán \textbf{gestor de casos}, \textbf{pacietne} y \textbf{cuidador principal}. Para tratar de desarrollar la aplicación nos centramos en dar solución a unos requisitos iniciales que les pareció interesante. 

En una primera fase vamos y para dar comienzo al proyecto se va a realizar el desarrollo e implementación de los siguientes puntos:
\begin{itemize}
\item \textbf{Administrador}
	\begin{itemize}
		\item \textbf{Gestión usuarios}
			\begin{itemize}
				\item \textbf{Gestión administradores}. Ver, invitar y eliminar
				\item \textbf{Gestión gestores de casos}. Ver, invitar y eliminar
				\item \textbf{Gestión pacientes}. Ver
				\item \textbf{Gestión cuidadores}. Ver
			\end{itemize}
		\item \textbf{Gestión roles}. Ver, crear y eliminar
		\item \textbf{Gestión hospitales}. Ver, crear y eliminar
		\item \textbf{Gestión tratamientos}. Ver, crear y eliminar
	\end{itemize}
\item \textbf{Gestor casos}
			\begin{itemize}
				\item \textbf{Gestión citas}. Crear, modificar y eliminar
				\item \textbf{Gestión pacientes}. . Ver, invitar y eliminar
			\end{itemize}
\item \textbf{Paciente}
			\begin{itemize}
				\item \textbf{Usuarios asociados}. Enviar invitación a cuidador
				\item \textbf{Gestión citas}. Ver y eliminar
			\end{itemize}
\item \textbf{Cuidador}
			\begin{itemize}
				\item \textbf{Gestión citas}. Ver y eliminar
			\end{itemize}
\end{itemize}

Una vez analizados los requisitos iniciales y ante el tamaño que supone el desarrollo completo de los requisitos, nos ponemos en contacto con la asociación informándoles que para una primera versión nos centraríamos la gestión de los actores que van a interactuar con la aplicación y la gestión de los datos que necesitan los usuarios para realizar la entrega de un proyecto funcional y que pueda ser utilizado por personal sanitario, paciente o cuidador. 
Bajo nuestro criterio es el más importante crear un producto que se pueda utilizar desde el principio y en futuras y siguientes versiones se iría completando con futuros desarrollos que irán haciendo más robusta y funcional nuestra aplicación.

Algunos ejemplos de futuras versiones son 
\begin{itemize}
\item \textbf{Material ortoprotésico}
			\begin{itemize}
				\item \textbf{Gestión material}.
			\end{itemize}
\item \textbf{Unidades ELA}
			\begin{itemize}
				\item \textbf{Gestión unidades ELA}.
			\end{itemize}
\item \textbf{Ensayos clínicos}
			\begin{itemize}
				\item \textbf{Gestión ensayos clínicos}
			\end{itemize}
\item \textbf{Artículos de Invesigación}
			\begin{itemize}
				\item \textbf{Gestión artículos investigación}
			\end{itemize}
\end{itemize}
Una vez realizada el análisis de los requisitos, nos informamos a cerca de que la \textbf{Esclerosis Lateral Amiotrófica (ELA)}~\cite{wiki:ela} es una enfermedad neuromuscular degenerativa que provoca de manera gradual la parálisis muscular y que provoca con el tiempo la muerte del paciente que padece dicha enfermedad.

Tras la investigación y analizando que actores inicialmente van a interactuar con el aplicativo obtenemos tres tipos de usuarios que serían el/la gestor/a de casos que se encargaría de agendar las citas para atender a los pacientes, los cuidadores principales que en un principio serían personas elegidas por los pacientes y finalmente los pacientes que padecen la enfermedad.


