\capitulo{6}{Trabajos relacionados}

La idea principal realizar una aplicación frontend haciendo uso del backend generado en el TFG \textbf{GII\_O\_MA\_22.05 Backend en microservicios para aplicación móvil}. A partir del uso del backend ya implementado se pretende crear una interfaz gráfica multiplataforma para usuarios afectados por la \textbf{Esclerosis Múltiple (EM)} y su posible extensión/adecuación para su uso en pacientes con la \textbf{Esclerosis Lateral Amiotrófica (ELA)}.

Ante los problemas técnicos observados durante el desarrollo de esta primera fase y bajo el consentimiento de \nomtutor, nos ponemos en contacto con la \textbf{Asociación de ELA de Castilla y León} y nos ponemos manos a la obra para crear una aplicación que inicialmente contará con una parte de administrador, que gestionará los usuarios, los hospitales, los tratamientos y roles. Así mismo, también crearemos una aplicación para los usuarios y que tendrá la funcionalidad de la gestión de citas. 

Algunas aplicaciones que están relacionadas son:
\begin{itemize}
\item \textbf{ME -- Multiple Esclerosis}. Para realizar un seguimiento y controlar sus síntomas.
\item \textbf{Cleo}. Centrada en el diálogo directo y con servicio de enfermería.
\item \textbf{Control EM}. También centrada en el control de la enfermedad.
\end{itemize}
