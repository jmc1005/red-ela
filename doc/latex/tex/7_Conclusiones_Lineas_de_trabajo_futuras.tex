\capitulo{7}{Conclusiones y Líneas de trabajo futuras}

Todo proyecto debe incluir las conclusiones que se derivan de su desarrollo. Éstas pueden ser de diferente índole, dependiendo de la tipología del proyecto, pero normalmente van a estar presentes un conjunto de conclusiones relacionadas con los resultados del proyecto y un conjunto de conclusiones técnicas. 

\section{Conclusiones}
Las conclusiones que sacamos durante el trayecto del desarrollo de las aplicaciones para el trabajo final de grado, son que una aplicación no siempre que se tiene una idea inicial se consigue llegar a termino y finalizarla exitosamente. Por fortuna, estamos en un mundo en el que tenemos que estar abierto a cambios y a enfrentarnos a retos para poder dar solución a los problemas de los usuarios finales.

A continuación, vamos a enumerar conclusiones obtenidas durante la realización del trabajo:
\begin{itemize}
\item Ante los inconvenientes durante el transcurso del proyecto, podemos afirmar que estamos contentos con el resultado obtenido y con el rumbo con el que hemos comenzado. Para empezar esta primera versión se centra en la gestión de las citas que tiene el usuario programadas en el tiempo.
\item El haber utilizado el framework Flutter para desarrollar la aplicación nos permite crear una aplicación multiplataforma siendo de código abierto, se integra bien con otras tecnologías, como en nuestro caso es Firebase y una vez adquiriendo la experiencia suficiente facilita el desarrollo.
\item Se emplean los conocimientos utilizados durante el grado para el desarrollo de la aplicación.
\end{itemize}

\section{Líneas de trabajo futuras}
Como en todas las aplicaciones se parte de una versión inicial y esta aplicación puede seguir creciendo para ir incluyendo nuevas funcionalidades y desarrollos que facilitarán el día a día de los usuarios.
\begin{itemize}
\item Crear opción para ver el material ortoprotésico
\item Crear opción para ver el listado de las unidades de ELA disponibles
\item Crear opción para ver el listado de las los ensayos clínicos
\item Crear opción para ver el listado de las artículos de investigación
\item Crear opción para mantener una conversación con la gestora de casos para agendar una cita.
\end{itemize}
