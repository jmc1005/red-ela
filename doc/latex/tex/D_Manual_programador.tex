\apendice{Documentación técnica de programación}

\section{Introducción}

Este anexo va orientado a describir y documentar la información necesaria sobre la poder realizar la instalación de las herramientas necesarias para el desarrollo, que estructura tiene la aplicación, como compilarlo, que servicios se necesitan.

- Cuenta firebase
- Generar dotenv
- Descargar el proyecto
- Comandos para arrancar el proyecto

\section{Estructura de directorios}
La estructura de nuestro proyecto está orientado en la arquitectura MVVM y será la siguiente;
\imagen{mvvm-clean-arquitecture}{mvvm}

\begin{itemize}
	\item \textcolor{darkblue}{\url{/.github}}: carpeta contiene los actions para los despliegues CI/CD en web y android al realizar push sobre la rama \textbf{main}.
	\item \textcolor{darkblue}{\url{/android}}: carpeta que contiene el código para la aplicación android.
	\item \textcolor{darkblue}{\url{/assets}}: carpeta que contendrá las imágenas y  fichero dotenv para las variables de entorno.
	\item \textcolor{darkblue}{\url{/doc}}: carpeta que contendrá la memoria y anexos.
	\item \textcolor{darkblue}{\url{/lib}}: directorio donde se encuentra el código fuente del proyecto.
		\begin{itemize}
			\item \textcolor{darkblue}{\url{main.dart}}: fichero principal del proyecto.
			\item \textcolor{darkblue}{\url{/app}}: contendrá la estructura de carpetas de nuestro Model-View-ViewModel
			\begin{itemize}
				\item \textcolor{darkblue}{\url{/data}}: contiene las implementaciones a los repositorios
				\item \textcolor{darkblue}{\url{/domain}}: contiene los modelos y las interfaces de los repositorios
				\item \textcolor{darkblue}{\url{/presententation}}: contiene los modulos que representan las vistas de nuestra aplicación
			\end{itemize}
			\item \textcolor{darkblue}{\url{/l10n}}: contrendrá los archivos para la internacionalización de la aplicación
		\end{itemize}
	\item \textcolor{darkblue}{\url{/web}}: carpeta que contiene el código para la aplicación web
	\item \textcolor{darkblue}{\url{analysis_options.yaml}}
	\item \textcolor{darkblue}{\url{build.yaml}}
	\item \textcolor{darkblue}{\url{pubspec.yaml}}: archivo que contendrá los paquetes necesarios para el correcto funcionamiento de la aplicación
	\item \textcolor{darkblue}{\url{l10n.yaml}}: archivo para la internacionalización de la aplicación
\end{itemize}


\section{Manual del programador}
Este manual está pensado para que futuros programadores puedan poner en marcha el proyecto y saber que pasos deben seguir para el correcto funcionamiento.

Cómo primer requisito es instalar Flutter~\cite{flutter}, para ello tendremos que ir a la página oficial de flutter \url{https://docs.flutter.dev/get-started/install} y elegir el sistema donde vamos a realizar la instalación y para que tipo de aplicación queremos desarrollar (en nuestro caso android).
\imagen{flutter-so}{Plataforma desarrollo}

Una vez realizada las instalaciones y configuraciones correspondientes, debemos comprobar que todo ha ido correctamente, para ello utilizaremos la siguiente instrucción en el terminal de \textbf{VS Code} \textbf{flutter doctor}
\imagen{flutter-doctor-command}{Comprobación correcta instalación Flutter}

\subsection{Visual Studio Code}
Para ello debemos de tener instalado VS Code, en caso contrario descargar de la página oficial \url{https://code.visualstudio.com/}.

Una vez instalado añadimos la extensión de Flutter
\imagen{flutter-extension}{Añadir Flutter a Visual Studio Code}

\subsection{Flutter SDK}
Siguiendo los pasos que indican en la guía de Flutter, nosotros escogimos instalar Flutter SDK desde VS Code, para ello debemos hacer:
\begin{enumerate}
	\item Abrir \textbf{VS Code}
	\item Pulsar \textbf{Ctrl}+\textbf{Shift}+\textbf{P} para abrir la paleta de comandos
	\item Seleccionar \textbf{New Project}
		\begin{itemize}
			\item Si no lo tienes descargado, pulsa en \textbf{Download SDK}.
			\item Si lo tienes descargado, pulsa \textbf{Locate SDK}
		\end{itemize}
	\item Cuando te pregunte que plantilla usar, hay que ignorarlo.
\end{enumerate}

\imagen{flutter-command-palette}{Paleta de comandos VS Code}

\subsection{Android Studio}
Para ello si no lo tenemos instalado, tendremos que realizar la instalación de \textbf{Android Studio} desde \url{https://developer.android.com/studio/install#windows} en mi caso lo instalo para Windows. Una vez instalado deberemos instalar el plugin de Flutter para Intellij \url{https://plugins.jetbrains.com/plugin/9212-flutter}

Para poder crear aplicación con Flutter, necesitamos comprobar que tenemos instalados los siguientes componentes de Android:
\begin{itemize}
	\item Android SDK Platform, API 34.0.5
	\item Android SDK Command-line Tools
	\item Android SDK Build-Tools
	\item Android SDK Platform-Tools
	\item Android Emulator
\end{itemize}

\subsubsection{Configuracion Android Studio}
Realizar la siguiente configuración de android studio.
\begin{enumerate}
	\item Habilite la aceleración de VM en su computadora de desarrollo.
	\item Inicie \textbf{Android Studio}.
	\item Vaya al diálogo de Configuración para ver el \textbf{Administrador de SDK}.
	\begin{itemize}
		\item Si tiene un proyecto abierto, vaya a \textbf{Herramientas} > \textbf{Administrador de dispositivos}.
		\item Si se muestra el diálogo de bienvenida, haga clic en el icono pulsar \textbf{más opciones} que sigue al botón \textbf{abrir} y seleccione \textbf{administrador de dispositivos} desde el menú desplegable.
	\end{itemize}
	\item Haga clic en \textbf{virtual}.
	\item Haga clic en \textbf{crear dispositivo}. El diálogo de configuración de dispositivo virtual se muestra.
	\item Seleccione el tipo de dispositivo bajo la categoría.
	\item Seleccione una definición de dispositivo. Puede explorar o buscar por el dispositivo.
	\item Haga clic en \textbf{siguiente}.
	\item Haga clic en \textbf{imagenes x86}.
	\item Haga clic en una imagen del sistema para la versión de Android que desee emular.
	\begin{itemize}
		\item Si la imagen deseada tiene un icono de descarga al lado del nombre de lanzamiento, haga clic en él.
		 \item Cuando complete la descarga, haga clic en \textbf{finalizar}.
	\end{itemize}
	\item Haga clic en \textbf{siguiente}. La configuración del dispositivo virtual se muestra en su paso de verificación.
	\item Para renombrar el \textbf{Dispositivo Virtual de Android (AVD)}, cambie el valor en el cuadro de texto \textbf{nombre AVD}.
	\item Haga clic en \textbf{mostrar configuraciones avanzadas} y desplace hasta \textbf{rendimiento emulado}.
	\item Desde el menú desplegable \textbf{gráficos}, seleccione \textbf{Hardware - GLES 2.0}. (Esto habilita la aceleración de hardware y mejora el rendimiento de renderizado).
	\item Verifique la configuración de su AVD. Si todo está correcto, haga clic en \textbf{finalizar}.
	\item En el diálogo del Administrador de dispositivos, haga clic en el icono \textbf{ejecutar} al lado derecho del AVD deseado. El emulador se inicia y muestra la pantalla predeterminada para la versión de Android OS y dispositivo seleccionados.
\end{enumerate}

\subsubsection{Fichero variables de entorno}
Antes de generar el fichero, debemos seguir los pasos descritos en el paquete de \textbf{Flutter} \textbf{encrypt} \url{https://pub.dev/packages/encrypt}:
\begin{itemize}
	\item Activar el paquete \fboxrule=1pt\fboxsep=4pt\fcolorbox{darkblue}{cgoLight}{pub global activate encrypt}
\end{itemize}

Para generar las claves seguras de manera aleatoria utilizaremos la siguiente instrucción \fboxrule=1pt\fboxsep=4pt\fcolorbox{darkblue}{cgoLight}{secure-random}. Con ella podremos generar claves que nos servirán para encriptar de manera segura los datos tratados por nuestra aplicación.
\imagen{flutter-encrypt-secure-random}{Encrypt secure random}

Este fichero lo tenemos que generar dentro de la carpeta \textbf{assets} con el nombre \textbf{dotenv}, ya que está ignorado en las subidas a \textbf{Github}. El fichero contendrá unas claves necesarias para el correcto funcionamiento de la aplicación:
\begin{itemize}
	\item \textbf{ENCRYPT\_KEY} esta clave la generaremos con la instrucción anterior y cuya longitud debe de ser de 32
	\item \textbf{ENCRYPT\_IV }esta clave la generaremos con la instrucción anterior y cuya longitud debe de ser de 12
	\item \textbf{USER\_EMAIL}=redelaubutfg@gmail.com
	\item \textbf{USER\_PASS}=qdmucyqwausbluzw
	\item \textbf{VAPID\_KEY} esta clave la necesitamos para poder obtener notificaciones push en nuestra web. Explicado en \textbf{\ref{subsec:firebase} Firebase}
\end{itemize}

\subsubsection{\label{subsec:firebase}Firebase}
Crearemos una cuenta Firebase y agregaremos un proyecto con el nombre \textbf{RedELA}
\begin{enumerate}
\item Agregaremos el proyecto 
\imagen{firebase-agregar-proyecto}{Agregar proyecto}
\item Introduciremos el nombre del proyecto. Pulsaremos \textbf{continuar}
\imagen{firebase-nombre-proyecto}{Agregar proyecto}
\item Deshabilitaremos Google Analytics y pulsaremos \textbf{crear proyecto}
\end{enumerate}

Tras unos segundos tendremos creado nuestro proyecto, ahora necesitamos enlazar ese proyecto con nuestro código descargado desde \textbf{\ref{subsec:github} Github} \url{https://github.com/jmc1005/red-ela}

Una vez creado nuestro proyecto en Firebase, debemos obtener el \textbf{VAPID\_KEY} en \textbf{descripción general} \textbf{configuración del proyecto} vamos a la sección de \textbf{cloud messaging} y al final en \textbf{configuración web} en \textbf{certificacdos push web} encontraremos en \textbf{par de claves} nuestro \textbf{VAPID\_KEY}.

\subsubsection{\label{subsec:github}Github}
Para empezar, necesitamos crear nuestros secrets en Github. Para acceder a \textbf{secrets} debemos a acceder a \textbf{Settings} y luego sobre el menú lateral izquierdo ir a \textbf{Secrets and variables} y pulsaremos sobre actions
\imagen{github-secrets}{Secrets and variables}

Posteriormente, sobre \textbf{Repository secrets} añadiremos nuestras variables necesarias para los despliegues mediante los workflow.
\imagen{github-repository-secrets}{Respository secrets}

En este punto vamos a explicar un poco a cerca de los workflows definidos en nuestro proyecto y además crear los secrets que serán necesarios para nuestro proyecto.

Los workflow son acciones que nos van a permitir desplegar en el hosting de firebase nuestra web y generar nuestras apks 

\subsubsection{Google Play}
De momento sólo estamos haciendo despliegues en plan Prueba Interna, para poder desplegar la aplicación para prueba interna necesitamos generar:
\begin{enumerate}
	\item Modificar nuestro fichero \textbf{pubspec\.yaml} cambiando la versión por uno más es decir, si tenemos \textbf{version: 1.0.3+3} poner \textbf{version: 1.0.4+4}
	\item un \textbf{bundle} con el comando \fboxrule=1pt\fboxsep=4pt\fcolorbox{darkblue}{cgoLight}{flutter build appbundle}, una vez generado el \textbf{bundle} accedemos a nuestra \textbf{consola de google play} \url{https://play.google.com/console/u/0/developers}
	\item pulsar en \textbf{crear nueva versión}
\imagen{google-play-nueva-version} {Nueva versión}
	\item Subir el bundle generado y actualizar la versión, por ejemplo a 4 (1.0.4)
\imagen{google-play-subir-bundle} {Subir bundle}
\imagen{google-play-detalle-version} {Detalle versión}
\end{enumerate}

\section{Compilación}
Para poder compilar el proyecto debemos ejecutar estos comandos \fboxrule=1pt\fboxsep=4pt\fcolorbox{darkblue}{cgoLight}{dart run build\_runner build} y \fboxrule=1pt\fboxsep=4pt\fcolorbox{darkblue}{cgoLight}{flutter pub get}. El primer comando nos generará el código automático de las clases que están con la anotación @freezed en nuestro proyecto. Estas anotaciones nos permiten generar código automático con pocas líneas de código.

\section{Ejecución del proyecto}
La aplicación se puede ejecutar tanto en la web como en un dispositivo móvil. Para ejecutar las pruebas podemos realizarlas tanto en un dispositivo móvil con Android que tenga API 34 o bien accediendo a la web.

También se pueden hacer pruebas en local desde VSCode \fboxrule=1pt\fboxsep=4pt\fcolorbox{darkblue}{cgoLight}{flutter run -d chrome} si queremos que se ejecute en chrome o bien seleccionando un dispositivo como se muestra en la imagen.
\imagen{flutter-device}{Seleccionar dispositivo}

\section{Pruebas del sistema}
Se ha generado una clase que testea mediante \textbf{flutter\_test} Firebase que será el gestione los datos que utiliza la aplicación.

Para poder ejecutar los test deberemos pulsar los iconos que serán un play
\imagen{flutter-test}{Test}

Cuando los tests pasan se verá de la siguiente manera.
\imagen{flutter-test-ok}{Test OK}
