\capitulo{2}{Objetivos del proyecto}

% Este apartado explica de forma precisa y concisa cuales son los objetivos que se persiguen con la realización del proyecto. Se puede distinguir entre los objetivos marcados por los requisitos del software a construir y los objetivos de carácter técnico que plantea a la hora de llevar a la práctica el proyecto.

Nuestro proyecto se centra en crear una aplicación móvil para ayudar y dar soporte a personal sanitario, pacientes con ELA o cuidadores principales. A continuación vamos a enumerar los objetivos del proyecto:
\begin{itemize}
\item Crear y desarrollar una aplicación multiplataforma amigable.
\item Dar soporte a los gestores de casos de pacientes con ELA con la gestión de las citas.
\item Dar soporte a personas con ELA con la visualización de las citas.
\item Dar soporte a los cuidadores de personas con ELA con la visualización de las citas.
\item Facilitar el día a día de los usuarios que usen la aplicación.
\end{itemize}

\section{Objetivos técnicos} 
\begin{itemize}
\item Desarrollar una aplicación multiplataforma para web y dispositivos Android con API 33 o superior (en futuras versiones implantable en iPhone).
\item Adquirir los conocimientos e información necesaria para realizar el proyecto con la arquitectura MVVM (Model-View-ViewModel)~\cite{mvvm} para el desarrollo de la aplicación.
\item Investigar sobre el framework Flutter~\cite{flutter} desarrollado por Google  y crear un proyecto con el resultado de una aplicación multiplataforma con código nativo para distintos dispositivos como en nuestro caso será para Android y web.
\item Con los conocimientos obtenidos durante el grado hacer uso de  GitHub para el control de versiones y el uso de actions para el despliegue CI/CD en la web y generar la aplicación Android.
\item Obtener la habilidad necesaria para hacer uso de la plataforma Firebase de la cual usaremos los servicios disponibles de base de datos, notificaciones, autenticación y hosting para alojar la web.
\item Obtener los recursos necesarios para el empleo de LaTeX~\cite{texmaker} para realizar la documentación de la memoria y los anexos. 
\item Obtener los conocimientos en Google Play para tener la posibilidad de distribuir la aplicación móvil. 
\end{itemize}

\section{Objetivos personales}
\begin{itemize}
\item Hacer uso de los conocimientos adquiridos durante el grado.
\item Investigar sobre el proyecto.
\item Emplear frameworks y herramientas novedosas, ágiles y óptimas.
\item Ampliar mis conocimientos en el mundo de la programación.
\end{itemize}