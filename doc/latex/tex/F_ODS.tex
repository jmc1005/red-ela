\apendice{Anexo de sostenibilización curricular}

\section{Introducción}
El objetivo del proyecto es crear una aplicación que facilite el día a día de los pacientes diagnosticados con ELA (Esclerósis Lateral Amiotrófica)~\cite{wiki:ela}. Esta aplicación está creada con el framework Flutter~\cite{flutter} y con la plataforma Firebase~\cite{firebase} que nos permite gestionar incialmente las citas y tratamientos que tiene le paciente. 

Además, con el desarrollo de la aplicación, se pretende concienciar y dar sostenibilidad al cuidado de personas con ELA haciendo que se comprometan aquellos profesionales de la salud y cuidadores principales ante los cuidados necesarios para hacer la vida más fácil a personas que hayan sido diagnosticadas con la enfermedad.

Para realizar el anexo de sostenibilidad se utiliza como referencia el documento de la CRUE~\cite{crue}

\subsection{Objetivos}
Algunos de los objetivos que podemos destacar son:
\begin{itemize}
\tightlist 
\item Concienciar la importancia sobre la sostenibilidad sobre los cuidados que necesitan los paciente con ELA.
\item Concienciar a crear soluciones sostenibles para el cuidado de pacientes con ELA.
\item Realizar desarrollos que promuevan la responsabilidad para dar solución a pacientes con ELA.
\end{itemize}

\subsection{Estrategias}
Algunas estrategias que se deben de usar son:
\begin{itemize}
\tightlist 
\item Realizar formaciones que fomenten la colaboración entre los profesionales de la salud, pacientes y cuidadores principales.
\item Sensibilizar y dar visibilidad a los pacientes que padezcan esta enfermedad.
\item Realizar desarrollos que promuevan la responsabilidad para dar solución a pacientes con ELA.
\end{itemize}

\subsection{Conclusión}
Sacamos como conclusión que la realización del proyecto aporta a la soluciones a la sociedad facilitando la vida a personas enfermas de ELA, a los cuidadores principales e incluso a los profesionales sanitarios. Además, a nivel medio ambiental reduce el uso de emisiones contaminantes ahorrando en el gasto de papel. 
Con ello se tendrá en cuenta la aceptación, demanda y aceptación por parte del usuario de la aplicación para realizar ajustes y nuevas funcionalidades a lo largo del tiempo