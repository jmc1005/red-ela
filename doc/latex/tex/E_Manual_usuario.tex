\apendice{Documentación de usuario}

\section{Introducción}
En este manual se indican que requisitos mínimos debe tener para poder utilizar la aplicación en un dispositivo móvil con sistema operativo Android 13 o superior.

\section{Requisitos de usuarios}
Los requisitos mínimos para hacer uso de la aplicación Android son:
\begin{itemize}
	\item Contar con un dispositivo Android 13 (Tiramisú – API 33) o superior.
\end{itemize}

Para acceder a la web se podrá hacer con cualquier dispositivo que contenga un navegador.

\section{Instalación}
Se podrá instalar la aplicación en un dispositivo móvil con Android desde \textbf{Google play} o a partir desde la apk activando el modo desarrollador en el dispositivo android.

\section{Manual del usuario}
Para empezar esta aplicación consta por un registro por invitación, es decir, que para que un usuario pueda realizar el registro debe ser invitado por otra persona. En un primer momento, las invitaciones serán las siguientes según los actores de la aplicación:
\begin{itemize}
	\item El \textbf{administrador} puede invitar a nuevos \textbf{administradores} o a \textbf{gestores de casos}
	\item El \textbf{gestor de casos} puede invitar a nuevos \textbf{gestores} o a \textbf{pacientes}
	\item El \textbf{paciente} puede invitar a un único \textbf{cuidador}
\end{itemize}

En la parte superior de la aplicación podremos actualizar y modificar nuestros datos o salir de la aplicación.
	\imagenTam{redela-appbar}{Acceder a RedELA}{.7}

Las pantallas de la aplicación serán:
\begin{itemize}
	\item Acceder mediante \textbf{correo electrónico} y \textbf{contraseña}
	\imagenTam{redela-acceder}{Acceder a RedELA}{.5}
	\item Registro. Para poder registrarse se necesita haber recibido un correo electrónico que permite registrarse inicialmente a partir de una \textbf{contraseña de un sólo uso}.
	\imagenTam{redela-registro}{Registro en RedELA}{.5}
	\item Home. En esta pantalla se verán las citas que se tendrán en el día.
	\imagenTam{redela-home}{Home}{.5}
	\item Citas. En esta parte de la aplicación podremos ver el calendario con las citas, podemos ver las citas en el calendario mensual o en el calendario de listas programadas.
	\imagenTam{redela-citas}{Citas}{.5}
	\item Usuarios asociados. En esta parte veremos los usuarios relacionados. Es decir, si accede un gestor de casos, verá todos los pacientes asignados a él, el en caso de ser un paciente, se verá la información del cuidador principal y del gestor de casos, y en caso de ser un cuidador principal, aparecerá la información del paciente relacionado con él. 
		\imagenTam{redela-asociados-gestor}{Pacientes asociados gestor de casos}{.5}
		\imagenTam{redela-asociados-paciente}{Cuidador y gestor de casos asociado a paciente}{.5}
		\imagenTam{redela-asociados-cuidador}{Paciente asociado a cuidador}{.5}
\end{itemize}

\section{Manual del administrador}
Una vez accedido como admin veremos la pantalla principal donde podrá gestionar la los usuarios, roles, hospitales y tratamientos. Además de poder salir de la aplicación.
\imagenTam{redela-admin-home}{Home Admin}{0.5}

\subsection{Gestión de Usuarios}
Inicialmente se concibe la aplicación para que un usuario en la \textbf{gestión del usuario} sólo pueda agregar a \textbf{administradores} o \textbf{gestores de casos}, ya que consideramos que la información de los pacientes por protección de datos es un tema que hay que proteger.

\subsection{Gestión de Roles}
El usuario podrá ver un listado con los roles creados, además de poder crear o eliminar algún rol.

\subsection{Gestión de Hospitales}
El usuario podrá ver un listado con los hospitales creados, además de poder crear o eliminar algún hospital.

\subsection{Gestión de Tratamientos}
El usuario podrá ver un listado con los tratamientos creados, además de poder crear o eliminar algún tratamiento.
