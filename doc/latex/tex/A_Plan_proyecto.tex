\apendice{Plan de Proyecto Software}

\section{Introducción}
La finalidad inicial de este proyecto era ofrecer la oportunidad para mejorar la calidad de vida de las personas que han sido diagnosticadas con algún tipo de Esclerósis Múltiple (EM). Para ello se empezó a desarrollar una aplicación frontend que de manera amigable que daría soporte a aquellas personas que hayan decido utilizarla y padezcan la enfermedad.

En este apartado se valorarán y analizarán las la planificación temporal así como el estudio de viabilidad.

\section{Planificación temporal Redem}
Para la planificación temporal se tendrán en cuenta el tiempo que conlleva la curva de aprendizaje para poner en marcha el proyecto, así como las distintas funcionalidades que se van implementando según las necesidades aportadas por el cliente.
A través del tiempo se realizan sprints para poder desarrollar y añadir las funcionalidades correspondientes al proyecto.

Este proyecto estará alojado en Github \url{https://github.com/jmc1005/redem}

\begin{itemize}
\tightlist
\item Se aplicó una estrategia de desarrollo a partir de (\emph{sprints}) mediante iteraciones incrementales.
\item Se realizan \emph{sprints} que conllevan poco tiempo para la creación de funcionalidades.
\item Al finalizar el \emph{sprint} correspondiente se incorpora a la release que posteriormente formará parte del producto.
\item Se realizaban reuniones con el cliente para revisar el producto. En estas revisiones se aportan mejoras/modificaciones.
\item Tras la obtención de requisitos de cliente, se planifican las nuevas tareas y que deciden en que \emph{sprint} estarán asignadas.
\item Cada tarea incluida se estima en tiempo en un tablero \emph{scrum}.
\item Cada tarea estimada se priorizan dentro del tablero \emph{scrum}.
\end{itemize}

\subsection{Sprint 0 (19/10/2023)}
Se realiza reunión con el tutor {\nomtutor} para asignación del proyecto para realizar la parte frontend de una aplicación
para dar soporte a usuarios diagnoticados con Esclerósis Múltiples (EM).

En ella se comenta una idea inicial sobre que herramientas se van a emplear y la selección del framework para crear el proyecto.

\subsection{Sprint 1 (27/10/2023)}
Se realiza una primera toma de contacto con el cliente con los requisitos iniciales. En ella nos transmite que está muy ilusionado en 
que queramos aceptar el reto y poder dar a luz un aplicación que podrá facilitar la vida de mucha gente.

Para los objetivos iniciales nos informa que está interesado en crear una aplicación móvil para personas que hayan sido diagnosticadas
con EM y así poder aportar una solución en su día a día.

Le indicamos que tenemos intención en realizar la aplicación en un lenguaje que podría utilizarse en distintos dispositivos así como podría 
ser usada en distintos Sistemas Operativos. 

Inicialmente nos indica que una versión inicial estaría disponible para dispositivos con sistema operativo Android.

\subsection{Sprint 2 (28/10/2023 al 04/02/2024)}
Se realiza una primera toma de contacto con el framework Flutter después de analizar que ventajas e inconvenientes podríamos encontrarnos
para realizar la aplicación móvil solicitada por el cliente.

Tras tomar la decisión se empieza una formación inicial en el lenguaje de programación Dart para entender su funcionamiento y poder realizar
el desarrollo del proyecto.

\subsection{Sprint 3 (05/10/2023 al 11/02/2024)}
Se realiza la creación del proyecto con el framework Flutter y con el lenguaje de programación Dart. Además, se añaden las configuraciones iniciales
y se crea la página inicial.

\subsection{Sprint 4 (12/02/2024 al 18/02/2024)}
Se añaden las páginas de de Login y Registro al proyecto.

\subsection{Sprint 5 (19/02/2024 al 25/02/2024)}
Crear página detalle del usuario y home.

\subsection{Sprint 6 (26/02/2024 al 10/03/2024)}
Crear aplicación responsive para varios dispositivos (web, tablet, móvil)
Corregir bug de idioma
Enviar correo al cliente para resolver dudas a cerca de los requisitos del usuario.

\subsection{Sprint 7 (11/03/2024 al 17/03/2024)}
Actualizar datos usuario
Actualizar documentación
Crear página Administrador (tras las inquietudes transmitidas al tutor sobre la inconsistencia de datos, nos aconseja que nos creemos una página para que el administrador pueda realizar la gestión de síntomas, tipos de esclerosis múltiples,...)

\section{Planificación temporal RedELA}
Una vez tomada la decisión de enfocar el trabajo y el proyecto una vez nos pusimos en contacto con la Asociación de Esclerosis Lateral Amiotrófica de Castilla y León y nos informaron de los requisitos que necesitarían para crear una aplicación que facilitara la vida a pacientes con ELA nos planificamos de manera similar al proyecto inicial.

Este proyecto estará alojado en Github \url{https://github.com/jmc1005/red-ela}

\begin{itemize}
\tightlist
\item Se aplicó una estrategia de desarrollo a partir de (\emph{sprints}) mediante iteraciones incrementales.
\item Se realizan \emph{sprints} que conllevan poco tiempo para la creación de funcionalidades.
\item Al finalizar el \emph{sprint} correspondiente se incorpora a la release que posteriormente formará parte del producto.
\item Se realizaban reuniones con el cliente para revisar el producto. En estas revisiones se aportan mejoras/modificaciones.
\item Tras la obtención de requisitos de cliente, se planifican las nuevas tareas y que deciden en que \emph{sprint} estarán asignadas.
\item Cada tarea incluida se estima en tiempo en un tablero \emph{scrum}.
\item Cada tarea estimada se priorizan dentro del tablero \emph{scrum}.
\end{itemize}

\subsection{Sprint 0 (08/04/2024 - 14/04/2024)}
Analizar cambio rumbo del TFG.
Reunión con Asociación ELACyL para presentar idea del proyecto.
Analizar documento con requisitos iniciales enviado por la asociación
\begin{itemize}
\tightlist
\item Actores que interactúan (Gestor/a de Casos, Paciente, Cuidador principal)
\item Secciones de la aplicación
\item Fases entrega
\end{itemize}

\subsection{Sprint 1 (15/04/2024 - 21/04/2024)}
\begin{itemize}
\tightlist
\item Crear proyecto red-ela
\item Añadir configuración inicial
\item Añadir internacionalización, en esta primera versión sólo estará disponible en español.
\item Crear página registro/acceso
\item Crear página administrador
\end{itemize}

\subsection{Sprint 2 (22/04/2024 - 28/04/2024)}
\begin{itemize}
\tightlist 
\item Crear servicio autenticación usuario
\item Crear servicio gestión usuario
\item Crear página listado de usuario para la gestión de usuarios
\item Crear página detalle usuario
\end{itemize}

\subsection{Sprint 3 (29/04/2024 - 05/05/2024)}
\begin{itemize}
\tightlist 
\item Analizar flujo de los actores Paciente, Cuidador principal y Enfermera Gestora de Casos
\item Actualizar campos Detalle usuario
\end{itemize}

\subsection{Sprint 4 (06/05/2024 - 12/05/2024)}
\begin{itemize}
\tightlist 
\item Daily cliente.
\item Actualizar campos Detalle usuario
\end{itemize}

\subsection{Sprint 5 (13/05/2024 - 26/05/2024)}
\begin{itemize}
\tightlist 
\item Analizar crear OTP (One Time Password) para el registro de la aplicación.
\item Analizar enviar email con código de invitación para acceder página OTP (One Time Password)
\item Crear página OTP (One Time Password)
\item Crear funcionalidad OTP (One Time Password) con firebase a través del móvil
\item Crear servicio cifrar datos usuarios
\end{itemize}

\subsection{Sprint 6 (22/05/2024 - 02/06/2024)}
\begin{itemize}
\tightlist 
\item Separar en pantalla Admin los usuarios en Gestores de Casos, Pacientes y Cuidadores.
\item Enviar email al invitar a usuarios a regitrarse en la aplicación.
\item Actualizar Gestor de Casos
\item Actualizar Cuidador
\item Añadir pantalla Home
\item Añadir pantalla Citas
\end{itemize}

\subsection{Sprint 7 (03/06/2024 - 16/06/2024)}
\begin{itemize}
\tightlist 
\item Crear servicio gestión de citas
\item Añadir funcionalidad cancelar cita
\item Actualizar pantalla home
\item Ajustes envío email/invitación.
\item Añadir notificaciones push
\end{itemize}

\subsection{Sprint 8 (17/06/2024 - 30/06/2024)}
\begin{itemize}
\tightlist 
\item Corrección bugs
\item Envío notificaciones
\item Crear/Actualizar workflows para despliegue android/web
\item Crear cuenta desarrollador para despliegue Android
\end{itemize}

\section{Estudio de viabilidad}

\subsection{Viabilidad económica}
En este apartado vamos a realizar un supuesto sobre el coste que tendría la realización del proyecto en una empresa.

\subsubsection{Tiempo de desarrollo}
Esta aplicación al ser personalizada y con funcionalidades complejas vamos a indicar que el tiempo de desarrollo completo de la primera versión serán de unos 6 meses.

\subsubsection{Costes Desarrollador}
Suponemos que la aplicación ha sido realizada por un desarrollador con contrato temporal durante 6 meses y se considera que los gastos ocasionados serían:
\begin{table}[H]
\centering
\begin{tabular}{lr}
\toprule
Concepto	&	Coste   \\
\midrule
Salario bruto & 1.833,33 €\\
Complementos & 0,00 €\\
Contingencias Comunes (23,60\%) & 432,67 €\\
Accidentes de Trabajo y Enfermedades Profesionales (2,25\%) & 41,25 €\\
Desempleo (6,70\%) & 122,83 € \\
Formación Profesional (0,60\%) & 11,00 € \\
Fondo de garantía salarial (FOGASA) (0,20\%) & 3,67 €\\
\textbf{Coste del empleado} & \textbf{2.444,75 €}\\
\bottomrule
\end{tabular}
\caption{Costes realizar aplicación por un desarrollador.}
\label{costesdesarrollador}
\end{table}

\subsubsection{Costes Material}
Los costes para que el desarrollador pueda realizar la aplicación y además, se tiene en cuenta que el material que necesita ronda una amortización durante 5 años, son los siguientes:
\begin{table}[H]
\centering
\begin{tabular}{lcc}
\toprule
Concepto	&	Coste	&	Coste Amortizado  \\
\midrule
Portátil & 1.400 €	& 140 € \\
Cuenta desarrollador & 25 € & --\\
Dispositivo móvil android & 280 € & 28 €\\
Firebase & coste por uso & --\\
\textbf{Total} & \textbf{1705 €} & \textbf{168 €}\\
\bottomrule
\end{tabular}
\caption{Costes del material o subscripciones para realizar aplicación.}
\label{costesmaterial}
\end{table}

\subsubsection{Monetización}
Esta aplicación va a ser una aplicación gratuita para ayudar a pacientes que hayan sido diagnosticadas con la enfermedad ELA, de momento será gratuita. Sería la Asociación de Esclerosis Lateral Amiotrófica de Castilla y León la que tendría que considerar la manera de sacarle rentabilidad a la aplicación. Algunas de las opciones podrían ser:
\begin{itemize}
\tightlist 
\item Donaciones
\item Publicidad
\item Crowdfunding
\end{itemize}

\subsection{Viabilidad legal}
Para definir la viabilidad legal~\cite{viabilidadLegal} de nuestro proyecto tenemos primeramente que comprender que tan importante es para nuestro desarrollo entender que hay que tener en cuenta lo siguiente:
\begin{itemize}
\tightlist 
\item Debe ajustarse a las leyes y regulaciones.
\item Afecta al coste, calidad, reputación e incluso a la duración.
\item Durante el desarrollo del proyecto este tendrá la participación y colaboración de diversos colaboradores, en nuestro caso el equipo del proyecto, el cliente.
\end{itemize}

Nuestra aplicación, al estar relacionada con la sanidad pública debemos tener cuidado con la privacidad del paciente.

\subsubsection{Software}
Para Flutter~\cite{flutter} que es un framework para construir aplicaciones multiplataforma, la viabilidad legal se centra en:
\begin{itemize}
\tightlist 
\item La licencia Apache 2.0, para crear y redistribuir la aplicación de código abierto.
\item La protección del framework mediante la propiedad intelectual de Google.
\item Derechos de autor.
\end{itemize}

Para Firebase~\cite{firebase} que fue creado por Google contiene algunos puntos a tener en cuenta:
\begin{itemize}
\tightlist 
\item Está creado bajo la licencia Apache 2.0, permitiendo a los desarrolladores poder utilizar, modificar y redistribuir sin tener alguna restricción y de manera gratuita.
\item La propiedad intelectual pertenece a Google.
\item Al crear la cuenta para utilizar la plataforma se aceptan los términos del servicio.
\item Los desarrolladores cuentan con los derechos de autor sobre su aplicación.
\item Los desarrolladores están obligados a cumplir las políticas de privacidad marcadas por Google.
\end{itemize}

