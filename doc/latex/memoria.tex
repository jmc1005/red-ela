\documentclass[a4paper,12pt,twoside]{memoir}

% Castellano
\usepackage[spanish,es-tabla]{babel}
\selectlanguage{spanish}
\usepackage[utf8]{inputenc}
\usepackage[T1]{fontenc}
\usepackage{lmodern} % Scalable font
\usepackage{microtype}
\usepackage{placeins}

\RequirePackage{booktabs}
\RequirePackage[table]{xcolor}
\RequirePackage{xtab}
\RequirePackage{multirow}

% Links
\PassOptionsToPackage{hyphens}{url}\usepackage[colorlinks]{hyperref}
\hypersetup{
	allcolors = {blue}
}

% Ecuaciones
\usepackage{amsmath}

% Rutas de fichero / paquete
\newcommand{\ruta}[1]{{\sffamily #1}}

% Párrafos
\nonzeroparskip

% Huérfanas y viudas
\widowpenalty100000
\clubpenalty100000

% Imágenes

% Comando para insertar una imagen en un lugar concreto.
% Los parámetros son:
% 1 --> Ruta absoluta/relativa de la figura
% 2 --> Texto a pie de figura
% 3 --> Tamaño en tanto por uno relativo al ancho de página
\usepackage{graphicx}
\newcommand{\imagen}[3]{
	\begin{figure}[!h]
		\centering
		\includegraphics[width=#3\textwidth]{#1}
		\caption{#2}\label{fig:#1}
	\end{figure}
	\FloatBarrier
}

% Comando para insertar una imagen sin posición.
% Los parámetros son:
% 1 --> Ruta absoluta/relativa de la figura
% 2 --> Texto a pie de figura
% 3 --> Tamaño en tanto por uno relativo al ancho de página
\newcommand{\imagenflotante}[3]{
	\begin{figure}
		\centering
		\includegraphics[width=#3\textwidth]{#1}
		\caption{#2}\label{fig:#1}
	\end{figure}
}

% El comando \figura nos permite insertar figuras comodamente, y utilizando
% siempre el mismo formato. Los parametros son:
% 1 --> Porcentaje del ancho de página que ocupará la figura (de 0 a 1)
% 2 --> Fichero de la imagen
% 3 --> Texto a pie de imagen
% 4 --> Etiqueta (label) para referencias
% 5 --> Opciones que queramos pasarle al \includegraphics
% 6 --> Opciones de posicionamiento a pasarle a \begin{figure}
\newcommand{\figuraConPosicion}[6]{%
  \setlength{\anchoFloat}{#1\textwidth}%
  \addtolength{\anchoFloat}{-4\fboxsep}%
  \setlength{\anchoFigura}{\anchoFloat}%
  \begin{figure}[#6]
    \begin{center}%
      \Ovalbox{%
        \begin{minipage}{\anchoFloat}%
          \begin{center}%
            \includegraphics[width=\anchoFigura,#5]{#2}%
            \caption{#3}%
            \label{#4}%
          \end{center}%
        \end{minipage}
      }%
    \end{center}%
  \end{figure}%
}

%
% Comando para incluir imágenes en formato apaisado (sin marco).
\newcommand{\figuraApaisadaSinMarco}[5]{%
  \begin{figure}%
    \begin{center}%
    \includegraphics[angle=90,height=#1\textheight,#5]{#2}%
    \caption{#3}%
    \label{#4}%
    \end{center}%
  \end{figure}%
}
% Para las tablas
\newcommand{\otoprule}{\midrule [\heavyrulewidth]}
%
% Nuevo comando para tablas pequeñas (menos de una página).
\newcommand{\tablaSmall}[5]{%
 \begin{table}
  \begin{center}
   \rowcolors {2}{gray!35}{}
   \begin{tabular}{#2}
    \toprule
    #4
    \otoprule
    #5
    \bottomrule
   \end{tabular}
   \caption{#1}
   \label{tabla:#3}
  \end{center}
 \end{table}
}

%
% Nuevo comando para tablas pequeñas (menos de una página).
\newcommand{\tablaSmallSinColores}[5]{%
 \begin{table}[H]
  \begin{center}
   \begin{tabular}{#2}
    \toprule
    #4
    \otoprule
    #5
    \bottomrule
   \end{tabular}
   \caption{#1}
   \label{tabla:#3}
  \end{center}
 \end{table}
}

\newcommand{\tablaApaisadaSmall}[5]{%
\begin{landscape}
  \begin{table}
   \begin{center}
    \rowcolors {2}{gray!35}{}
    \begin{tabular}{#2}
     \toprule
     #4
     \otoprule
     #5
     \bottomrule
    \end{tabular}
    \caption{#1}
    \label{tabla:#3}
   \end{center}
  \end{table}
\end{landscape}
}

%
% Nuevo comando para tablas grandes con cabecera y filas alternas coloreadas en gris.
\newcommand{\tabla}[6]{%
  \begin{center}
    \tablefirsthead{
      \toprule
      #5
      \otoprule
    }
    \tablehead{
      \multicolumn{#3}{l}{\small\sl continúa desde la página anterior}\\
      \toprule
      #5
      \otoprule
    }
    \tabletail{
      \hline
      \multicolumn{#3}{r}{\small\sl continúa en la página siguiente}\\
    }
    \tablelasttail{
      \hline
    }
    \bottomcaption{#1}
    \rowcolors {2}{gray!35}{}
    \begin{xtabular}{#2}
      #6
      \bottomrule
    \end{xtabular}
    \label{tabla:#4}
  \end{center}
}

%
% Nuevo comando para tablas grandes con cabecera.
\newcommand{\tablaSinColores}[6]{%
  \begin{center}
    \tablefirsthead{
      \toprule
      #5
      \otoprule
    }
    \tablehead{
      \multicolumn{#3}{l}{\small\sl continúa desde la página anterior}\\
      \toprule
      #5
      \otoprule
    }
    \tabletail{
      \hline
      \multicolumn{#3}{r}{\small\sl continúa en la página siguiente}\\
    }
    \tablelasttail{
      \hline
    }
    \bottomcaption{#1}
    \begin{xtabular}{#2}
      #6
      \bottomrule
    \end{xtabular}
    \label{tabla:#4}
  \end{center}
}

%
% Nuevo comando para tablas grandes sin cabecera.
\newcommand{\tablaSinCabecera}[5]{%
  \begin{center}
    \tablefirsthead{
      \toprule
    }
    \tablehead{
      \multicolumn{#3}{l}{\small\sl continúa desde la página anterior}\\
      \hline
    }
    \tabletail{
      \hline
      \multicolumn{#3}{r}{\small\sl continúa en la página siguiente}\\
    }
    \tablelasttail{
      \hline
    }
    \bottomcaption{#1}
  \begin{xtabular}{#2}
    #5
   \bottomrule
  \end{xtabular}
  \label{tabla:#4}
  \end{center}
}



\definecolor{cgoLight}{HTML}{EEEEEE}
\definecolor{cgoExtralight}{HTML}{FFFFFF}

%
% Nuevo comando para tablas grandes sin cabecera.
\newcommand{\tablaSinCabeceraConBandas}[5]{%
  \begin{center}
    \tablefirsthead{
      \toprule
    }
    \tablehead{
      \multicolumn{#3}{l}{\small\sl continúa desde la página anterior}\\
      \hline
    }
    \tabletail{
      \hline
      \multicolumn{#3}{r}{\small\sl continúa en la página siguiente}\\
    }
    \tablelasttail{
      \hline
    }
    \bottomcaption{#1}
    \rowcolors[]{1}{cgoExtralight}{cgoLight}

  \begin{xtabular}{#2}
    #5
   \bottomrule
  \end{xtabular}
  \label{tabla:#4}
  \end{center}
}



\graphicspath{ {./img/} }

% Capítulos
\chapterstyle{bianchi}
\newcommand{\capitulo}[2]{
	\setcounter{chapter}{#1}
	\setcounter{section}{0}
	\setcounter{figure}{0}
	\setcounter{table}{0}
	\chapter*{\thechapter.\enskip #2}
	\addcontentsline{toc}{chapter}{\thechapter.\enskip #2}
	\markboth{#2}{#2}
}

% Apéndices
\renewcommand{\appendixname}{Apéndice}
\renewcommand*\cftappendixname{\appendixname}

\newcommand{\apendice}[1]{
	%\renewcommand{\thechapter}{A}
	\chapter{#1}
}

\renewcommand*\cftappendixname{\appendixname\ }

% Formato de portada
\makeatletter
\usepackage{xcolor}
\newcommand{\tutor}[1]{\def\@tutor{#1}}
\newcommand{\course}[1]{\def\@course{#1}}
\definecolor{cpardoBox}{HTML}{E6E6FF}
\definecolor{mygray}{gray}{0.9}

\def\maketitle{
  \null
  \thispagestyle{empty}
  % Cabecera ----------------
\noindent\includegraphics[width=\textwidth]{cabecera}\vspace{1cm}%
  \vfill
  % Título proyecto y escudo informática ----------------
  \colorbox{cpardoBox}{%
    \begin{minipage}{.8\textwidth}
      \vspace{.5cm}\Large
      \begin{center}
      \textbf{TFG del Grado en Ingeniería Informática}\vspace{.6cm}\\
      \textbf{\LARGE\@title{}}
      \end{center}
      \vspace{.2cm}
    \end{minipage}

  }%
  \hfill\begin{minipage}{.20\textwidth}
    \includegraphics[width=\textwidth]{escudoInfor}
  \end{minipage}
  \vfill
  % Datos de alumno, curso y tutores ------------------
  \begin{center}%
  {%
    \noindent\LARGE
    Presentado por \@author{}\\ 
    en Universidad de Burgos --- \@date{}\\
    Tutor: \@tutor{}\\
  }%
  \end{center}%
  \null
  \cleardoublepage
  }
\makeatother

\newcommand{\nombre}{Javier Martín Castro} %%% cambio de comando
\newcommand{\dni}{24392279C} %%% cambio de comando
\newcommand{\nomtutor}{Pedro Renedo Fernández} %%% cambio de comando
\newcommand{\nomcotutor}{} %%% cambio de comando
\newcommand{\titulo}{RedELA Multiplataforma } %%% cambio de comando
\newcommand{\departamentotutor}{Ingeniería Informática} %%% cambio de comando
\newcommand{\area}{Lenguajes y Sistemas Informáticos} %%% cambio de comando

% Datos de portada
\title{\titulo}
\author{\nombre}
\tutor{\nomtutor}

\date{\today}

\setlrmarginsandblock{3.5cm}{2.5cm}{*}
\setulmarginsandblock{2.5cm}{*}{1}
\checkandfixthelayout 

\begin{document}

\maketitle


\newpage\null\thispagestyle{empty}\newpage


%%%%%%%%%%%%%%%%%%%%%%%%%%%%%%%%%%%%%%%%%%%%%%%%%%%%%%%%%%%%%%%%%%%%%%%%%%%%%%%%%%%%%%%%
\thispagestyle{empty}


\noindent\includegraphics[width=\textwidth]{cabecera}\vspace{1cm}

\noindent D. \nomtutor, profesor del departamento de \departamentotutor, área de \area.

\noindent Expone:

\noindent Que el alumno D. \nombre, con DNI  \dni, ha realizado el Trabajo final de Grado en Ingeniería Informática titulado \titulo de TFG. 

\noindent Y que dicho trabajo ha sido realizado por el alumno bajo la dirección del que suscribe, en virtud de lo cual se autoriza su presentación y defensa.

\begin{center} %\large
En Burgos, {\large \today}
\end{center}

\vfill\vfill\vfill

% Author and supervisor
\begin{minipage}{0.45\textwidth}
\begin{flushleft} %\large
Vº. Bº. del Tutor:\\[2cm]
D. {\nomtutor}\\
\end{flushleft}
\end{minipage}
\hfill
\begin{minipage}{0.45\textwidth}
\begin{flushleft} %\large
% Vº. Bº. del co-tutor:\\[2cm]
% D. {\nomcotutor}\\
\end{flushleft}
\end{minipage}
\hfill

\vfill

% para casos con solo un tutor comentar lo anterior
% y descomentar lo siguiente
%Vº. Bº. del Tutor:\\[2cm]
%D. nombre tutor


\newpage\null\thispagestyle{empty}\newpage




\frontmatter

% Abstract en castellano
\renewcommand*\abstractname{Resumen}
\begin{abstract}

La idea principal de realizar este proyecto nace de poder facilitar la vida a personas con esclerosis~\cite{wiki:esclerosis} siendo una enfermedad autoinmune que se caracteriza por causar deterioro en el sistema nervioso. Esta enfermedad se puede manifestar como \textbf{esclerosis múltiple (EM)}~\cite{wiki:em} o \textbf{esclerosis lateral amiotrófica (ELA)}~\cite{wiki:ela}.

Este tipo de pacientes ya sea con \textbf{EM} o \textbf{ELA}, necesita una serie de atenciones y cuidados para que su día a día mejore dentro de la gravedad de la enfermedad y nuestra idea es crear una aplicación para gestionar las citas con profesionales de la salud. Con todo eso y finalmente para el desarrollo de la aplicación se cuenta con el asesoramiento de la \textbf{Asociación de ELA de Castilla y León (ELACyL)}. Los objetivos de esta asociación es dar visibilidad a aquellas personas afectadas para que la sociedad tome conciencia de la gravedad de la enfermedad, obtener más recursos y financiación para realizar investigaciones de la ELA y finalmente atender a los afectados por la enfermedad ya que necesitan una cantidad elevada de recursos para poder llevar una vida digna.

Para ayudar a estas personas se va a proceder a desarrollar una aplicación multiplataforma para la web y para dispositivos Android, en futuras versiones podrá desplegarse en dispositivos móviles Apple. Esta aplicación contará con un gestor de citas para que las personas puedan tener de manera amigable y cómoda todas las citas agendadas por el Gestor de casos.

\end{abstract}

\renewcommand*\abstractname{Descriptores}
\begin{abstract}
Esclerósis, Esclerosis Múltiple, Esclerosis Lateral Amiotrófica, ELA, ayuda, aplicación móvil Android, aplicación web, citas, pacientes, cuidadores principales.
\end{abstract}

\clearpage

% Abstract en inglés
\renewcommand*\abstractname{Abstract}
\begin{abstract}
The main idea behind this project is to facilitate the lives of individuals with sclerosis, an autoimmune disease that causes damage to the nervous system. This condition can manifest as \textbf{Multiple Sclerosis (MS)}~\cite{wiki:em} or \textbf{Amyotrophic Lateral Sclerosis (ALS)}~\cite{wiki:ela}.

Patients with \textbf{MS} or \textbf{ALS} require a range of care and attention to improve their daily lives within the severity of the disease. Our idea is to develop an application to manage appointments with healthcare professionals. With this goal in mind, we will be advised by the \textbf{ALS Association of Castilla and León (ELACyL)}. The objectives of this association are to raise awareness among those affected, to obtain more resources and funding for ALS research, and to provide support to those affected by the disease, as they require a significant amount of resources to live a dignified life.

To help these individuals, we will proceed to develop a multi-platform application for the web and Android devices, with future versions planned for deployment on Apple devices. This application will feature an appointment manager that allows individuals to easily and comfortably manage all scheduled appointments through a user-friendly interface.

Note: I translated "Asociación de ELA de Castilla y León" as "ALS Association of Castile and León", assuming that ELA stands for Amyotrophic Lateral Sclerosis, as mentioned earlier. If that's not the case, please let me know and I'll be happy to adjust the translation accordingly.
\end{abstract}

\renewcommand*\abstractname{Keywords}
\begin{abstract}
Sclerosis, Multiple Sclerosis, Amyotrophic Lateral Sclerosis, ALS, help, Android mobile app, web app, dating, patients, primary caregivers.
\end{abstract}

\clearpage

% Indices
\tableofcontents

\clearpage

\listoffigures

\clearpage

\listoftables
\clearpage

\mainmatter
\capitulo{1}{Introducción}

Incialmente el proyecto nacía para dar solución y mejorar la calidad de vida de personas que tengan \textbf{Esclerosis Múltiple}~\cite{wiki:em,medlineplus:em} (\textbf{EM}). Para ello se iba a desarrollar el Frontend inicialmente para dispositivo móvil Android. 

Resumiendo, podemos decir que la EM es una enfermedad neurológica crónica cuya naturaleza es inflamatoria y autoinmunitaria. Se caracteriza por la afectar el cerebro y la médula espinal y hace más lentos o bloquea la conexión entre el cerebro y el cuerpo.
Algunos síntomas de EM pueden ser:
\begin{itemize}
\item Alteraciones en la visión.
\item Debilidad a nivel muscular.
\item Problemas con la coordinación y el equilibrio.
\item Apacición de lesiones cerebrales detectadas en la resonancia magnética.
\end{itemize}
Podemos clasificar esta enfermedad entre:
\begin{itemize}
\item \textbf{Recurrente-remitente (EMRR)} caracterizada por aparición periódica de ataques o brotes.
\item \textbf{Secundaria progresiva (EMSP)} se caracteriza por el empeoramiento gradual y progresiva de la discapacidad.
\item \textbf{Primaria progresiva (EMPP)} este tipo comienza de manera inofensiva y empeora con el tiempo.
\item \textbf{Progresiva-recurrente (EMPR)} este es el más agresivo ya que las apariciones de los síntomas son severas y periódicas.
\end{itemize}

Ante los problemas de diseño para poder desarrollar la aplicación y con la aprobación del tutor, {\nomtutor}, nos ponemos en contacto con al asociación de ELA de Castilla y León  y realizamos una primera toma de contacto con ellos para saber que objetivos iniciales les parecería interesante para crear una aplicación multiplataforma que pudiera ayudar a pacientes o personas afectadas con la enfermedad. Al mismo tiempo, investigamos un poco de que trata la enfermedad y podemos afirmar que, la ELA que es una enfermedad degenerativa que daña el sistema nervioso periférico y central. Va degenerando de manera progresiva las células motoras del cerebro y la médula espinal.

Tras analizar los objetivos marcados por la asociación y ver que la aplicación contará inicialmente con tres actores que serán \textbf{gestor de casos}, \textbf{pacietne} y \textbf{cuidador principal}. Para tratar de desarrollar la aplicación nos centramos en dar solución a unos requisitos iniciales que les pareció interesante. 

En una primera fase vamos y para dar comienzo al proyecto se va a realizar el desarrollo e implementación de los siguientes puntos:
\begin{itemize}
\item \textbf{Administrador}. Este rol tiene la posibilidad de realizar las siguientes gestiones:
	\begin{itemize}
		\item \textbf{Gestión usuarios}
			\begin{itemize}
				\item \textbf{Gestión administradores}. Esta gestión se define por poder gestionar aquellos usuarios que son \textbf{administradores} de la aplicación. En ella se podrán ver la información relativa con los usuarios administradores y se podrá invitar a nuevos administradores. 
				\item \textbf{Gestión gestores de casos}. Esta gestión se define por poder gestionar aquellos usuarios que son \textbf{gestores de casos} de la aplicación. En ella se podrán ver la información relativa con los usuarios gestores de casos y se podrá invitar a nuevos gestores. 
				\item \textbf{Gestión pacientes}. Esta gestión se define por poder gestionar aquellos usuarios que son \textbf{pacientes} de la aplicación. En ella se podrán ver la información relativa con los usuarios que son pacientes. 
				\item \textbf{Gestión cuidadores}. Esta gestión se define por poder gestionar aquellos usuarios que son \textbf{gestores de casos} de la aplicación. En ella se podrán ver la información relativa con los usuarios cuidadores. 
			\end{itemize}
		\item \textbf{Gestión roles}. Aquí realizamos la gestión de los roles que va a tener la aplicación. Inicialmente nos encontramos con los roles de \textbf{administrador}, \textbf{gestor de casos}, \textbf{paciente} y \textbf{cuidador}.
		\item \textbf{Gestión hospitales}. Aquí realizamos gestiones pudiendo dar de alta, modificar o dar de baja hospitales.
		\item \textbf{Gestión tratamientos}. Aquí realizamos gestiones pudiendo dar de alta, modificar o dar de baja tratamientos.
	\end{itemize}
\item \textbf{Gestor casos}. Este rol puede realizar las siguientes operaciones en la aplicación:
			\begin{itemize}
				\item \textbf{Gestión citas}. El gestor de casos puede crear, modificar la cita o cancelar una cita para un paciente con ELA. Una vez realizada la operación de crear, cancelar o modificar una cita, llegará una notificación al paciente con la información.
				\item \textbf{Gestión pacientes}. El gestor de casos puede invitar, ver la información del paciente del cual gestiona el caso o eliminar un paciente fallecido para realizar el derecho al olvido.
			\end{itemize}
\item \textbf{Paciente}
			\begin{itemize}
				\item \textbf{Gestión citas}. En este caso el paciente podrá ver aquellas citas que tenga asociadas y cancelarlas. Una vez cancelada la cita el gestor de casos recibirá una notificación push.
				\item \textbf{Usuarios asociados}. El usuario al registrarse, en la pantalla de usuarios asociados, si no tiene cuidador asignado, podrá enviarle una invitación para que el usuario se registre. También puede ver en esta pantalla la información del gestor de su caso.
			\end{itemize}
\item \textbf{Cuidador}
			\begin{itemize}
				\item \textbf{Gestión citas}. En este caso el cuidador podrá ver aquellas citas que tenga asociadas el paciente del que se hace cargo y puede cancelarlas. Una vez cancelada la cita el gestor de casos recibirá una notificación push.
				\item \textbf{Usuarios asociados}. En esta pantalla verá la información del paciente que tenga asignado.
			\end{itemize}
\end{itemize}

Una vez analizados los requisitos iniciales y ante el tamaño que supone el desarrollo completo de los requisitos, nos ponemos en contacto con la asociación informándoles que para una primera versión nos centraríamos la gestión de los actores que van a interactuar con la aplicación y la gestión de los datos que necesitan los usuarios para realizar la entrega de un proyecto funcional y que pueda ser utilizado por personal sanitario, paciente o cuidador. 
Bajo nuestro criterio es el más importante crear un producto que se pueda utilizar desde el principio y en futuras y siguientes versiones se iría completando con futuros desarrollos que irán haciendo más robusta y funcional nuestra aplicación.

Algunos ejemplos de futuras versiones son 
\begin{itemize}
\item \textbf{Material ortoprotésico}
			\begin{itemize}
				\item \textbf{Gestión material}. En este punto se tratará de aquel material que sea de ayuda para facilitar el día a día de los paciente con ELA, como pueden ser gruas o camas.
			\end{itemize}
\item \textbf{Unidades ELA}
			\begin{itemize}
				\item \textbf{Gestión unidades ELA}. Aquí el usuario podrá ver aquellos puntos donde se encuentren las unidades de ELA para poder acercarse y consultar o recibir ayuda.
			\end{itemize}
\item \textbf{Ensayos clínicos}
			\begin{itemize}
				\item \textbf{Gestión ensayos clínicos}. Se almacenaran y podrán consultar aquellos ensayos clínicos relacionados con la esclerosis.
			\end{itemize}
\item \textbf{Artículos de Invesigación}
			\begin{itemize}
				\item \textbf{Gestión artículos investigación}. Se almacenaran y podrán consultar aquellos artículos de investigación relacionados con la esclerosis.
			\end{itemize}
\end{itemize}
Una vez realizada el análisis de los requisitos, nos informamos a cerca de que la \textbf{Esclerosis Lateral Amiotrófica (ELA)}~\cite{wiki:ela} es una enfermedad neuromuscular degenerativa que provoca de manera gradual la parálisis muscular y que provoca con el tiempo la muerte del paciente que padece dicha enfermedad.

Tras la investigación y analizando que actores inicialmente van a interactuar con el aplicativo obtenemos tres tipos de usuarios que serían el/la gestor/a de casos que se encargaría de agendar las citas para atender a los pacientes, los cuidadores principales que en un principio serían personas elegidas por los pacientes y finalmente los pacientes que padecen la enfermedad.



\capitulo{2}{Objetivos del proyecto}

% Este apartado explica de forma precisa y concisa cuales son los objetivos que se persiguen con la realización del proyecto. Se puede distinguir entre los objetivos marcados por los requisitos del software a construir y los objetivos de carácter técnico que plantea a la hora de llevar a la práctica el proyecto.

Nuestro proyecto se centra en crear una aplicación móvil para ayudar y dar soporte a personal sanitario, pacientes con ELA o cuidadores principales. A continuación vamos a enumerar los objetivos del proyecto:
\begin{itemize}
\item Crear y desarrollar una aplicación multiplataforma amigable.
\item Dar soporte a los gestores de casos de pacientes con ELA con la gestión de las citas.
\item Dar soporte a personas con ELA con la visualización de las citas.
\item Dar soporte a los cuidadores de personas con ELA con la visualización de las citas.
\item Facilitar el día a día de los usuarios que usen la aplicación.
\end{itemize}

\section{Objetivos técnicos} 
\begin{itemize}
\item Desarrollar una aplicación multiplataforma para web y dispositivos Android con API 33 o superior (en futuras versiones implantable en iPhone).
\item Adquirir los conocimientos e información necesaria para realizar el proyecto con la arquitectura MVVM (Model-View-ViewModel)~\cite{mvvm} para el desarrollo de la aplicación.
\item Investigar sobre el framework Flutter~\cite{flutter} desarrollado por Google  y crear un proyecto con el resultado de una aplicación multiplataforma con código nativo para distintos dispositivos como en nuestro caso será para Android y web.
\item Con los conocimientos obtenidos durante el grado hacer uso de  GitHub para el control de versiones y el uso de actions para el despliegue CI/CD en la web y generar la aplicación Android.
\item Obtener la habilidad necesaria para hacer uso de la plataforma Firebase de la cual usaremos los servicios disponibles de base de datos, notificaciones, autenticación y hosting para alojar la web.
\item Obtener los recursos necesarios para el empleo de LaTeX~\cite{texmaker} para realizar la documentación de la memoria y los anexos. 
\item Obtener los conocimientos en Google Play para tener la posibilidad de distribuir la aplicación móvil. 
\end{itemize}

\section{Objetivos personales}
\begin{itemize}
\item Hacer uso de los conocimientos adquiridos durante el grado.
\item Investigar sobre el proyecto.
\item Emplear frameworks y herramientas novedosas, ágiles y óptimas.
\item Ampliar mis conocimientos en el mundo de la programación.
\end{itemize}
\capitulo{3}{Conceptos teóricos}

\section{Herramientas proyecto}
\label{herramientasportipodeuso}
Para este proyecto no se emplean los conceptos teorícos, ya que la aplicación se desarrolla para facilitar la vida del paciente teniendo a mano las próximas citas médicas que tenga. \\

A continuación y mediante la tabla se van a mostrar las herramientas que se han empleado para llevar a cabo el proyecto. \\

Las herramientas usadas para el desarrollo del proyecto son:
\vspace*{-1cm}
\begin{table}
	 \rowcolors {2}{gray!35}{}
	 \vspace*{-2.5cm}
	 \begin{center}
		\begin{tabular}{l c c c c c}
			\toprule
				Herramientas & App Android & App Web & API REST & BD & Memoria   \\
			\midrule
				HTML5 & & X & & & \\
				CSS3 & & X & & &\\
				BOOTSTRAP & & X & & &\\
				JavaScript & & X & & &\\
				Flutter & X & X & X & X &\\
				Dart & X & X & X & X &\\
				FlutterFlow & X & X & X & X &\\
				Firebase & X & X & X & X &\\
				Github & X & X & & &\\
				Google Play & & X & & &\\
				\TeX{}Maker & & & &  & X\\
				VersionOne & X & X & X & X & X\\
			\bottomrule
		\end{tabular}
	\end{center}
	\caption{Herramientas y tecnologías utilizadas en cada parte del proyecto}
	\label{herramientasportipodeuso}
	\vspace*{3cm}
\end{table}
\capitulo{4}{Técnicas y herramientas}
\section{Metodologías}
\subsection{Scrum}
Scrum~\cite{wiki:scrum} es un marco que se emplea para la gestión de proyectos de metodología ágil y que facilita a los equipos la tarea de gestionar y estructurar el trabajo. Para ello, a través de iteraciones (sprints) se realiza una serie de tareas en un periodo de tiempo y se van incorporando al software.
\imagen{scrum}{Scrum}{.9}

\subsection{Gitflow}
Gitflow~\cite{gitflow} se define como un flujo de trabajo para la creación de ramas en Git y llevar un control de versiones. En este flujo existen dos ramas principales, master y develop, y otras ramas que pueden ayudarnos en nuestro desarrollo como pueden ser: feature, release o  hotfix.
\imagen{gitflow}{Gitflow}{.9}

\section{Patrón de diseño}
\subsection{Model-View-ViewModel MVVM}
Es un patrón MVVM~\cite{mvvm}, también conocido como Model View ViewModel que se centra en separar la interfaz del usuario (View) de la parte lógica (Model). La interacción entre la parte lógica y la interfaz del usuario a través del recurso ViewModel.

Algunas de las ventajas al usar este patrón son:
\begin{itemize}
\item Fácil desarrollo ya que al poder separar la vista de la lógica varios equipos pueden trabajar simultáneamente en varios componentes.
\item Fácil testeo ya que no es necesario utilizar la vista para crear tests para el model o el viewmodel.
\item Fácil mantenimiento ya que al realizar la separación de los componentes se crea un código simple y limpio.
\end{itemize}

Vamos a describir a continuación cada uno de los componentes que forman el MVVM:
\begin{itemize}
	\item Model: es el componente donde se encapsulan los datos de nuestra aplicación. En ella se pueden encontrar validación y lógica de negocio.
	\item View: nos muestra el diseño y la apariencia de nuestra aplicación. En ella se verán los datos pero sin contener nada de la lógica de negocio.
	\item ViewModel: es el componente que enlaza los datos o cambios de estado que puede tener nuestra aplicación.
\end{itemize}

\imagen{mvvm-pattern}{Arquitectura limpia}{.9}

\section{Repositorio}
Entre las herramientas consideradas GitHub~\cite{github}, Bitbucket~\cite{bitbucket}, GitLab~\cite{wiki:gitlab} se decide utilizar GitHub, porque nos permite alojar proyectos gratuitamente. Además podemos crear documentación a través de wikis, crear tareas, sprints.
Al crear el respositorio en GitHub podemos usar el gestor de proyecto que tiene a través de sus paneles y saber el estado en el que está cada issue (tarea).
Otra de las ventajas que encontramos con GitHub es que está integrado en multiples servicios de integración continua.

\section{Control de versiones}
Al haber escogido como repositorio GitHub para alojar nuestro proyecto, utilizamos como control de versiones Git~\cite{wiki:git} que es un sistema distribuido que nos permite realizar nuestro trabajo sin conexión, es ligero y rápido.
El crear ramas y hacer los merges es muy rápido y pocas veces se obtienen conflictos.
El historial es muy detallado.

\section{Gestión del proyecto}
Como gestor de proyectos utilizamos el que viene integrado en GitHub. Con esta herramienta podemos gestionar nuestro proyecto y ver en que estado está cada tarea, incluso podemos darle prioridad.

\section{Entorno de desarrollo integrado (IDE)}
\subsection{Flutter}
\imagen{flutter}{Flutter}{.9}
Flutter~\cite{flutter} es un SDK (Kit de Desarrollo de Software) desarrollado por Google que nos permite crear aplicaciones para varios dispositivos. Hemos elegido esta herramienta porque nos permite crear aplicaciones multiplataforma.
Entre alguna de las ventajas destacamos:
\begin{itemize}
\item \textbf{Compila en nativo} tanto para Android como iOS.
\item Mediante el uso de widgets, se pueden \textbf{crear interfaces gráficas flexibles}.
\item \textbf{Rapidez} en el desarrollo.
\end{itemize}

\subsection{Dart}
Dart~\cite{dart} es un \textbf{lenguaje de programación moderno}, desarrollado por Google, donde se mezcla la programación orientada a objetos POO con los lenguajes de programación basados en scripts.

\subsection{Firebase}
\imagen{firebase}{Firebase}{.9}
Firebase~\cite{firebase} es una plataforma que nos permite crear aplicaciones web y móviles. Nos permite utilizar un número de herramientas o servicios para facilitarnos a los desarrolladores construir aplicación de forma rápida, segura y escalable. En nuestro proyecto hemos empleado los servicios:
\begin{itemize}
\item Autenticación para la gestión de los usuarios.
\item Base de datos en la nube con Cloud Firestore que nos facilita la sincronización de los datos almacenados en tiempo real. Hemos decidido emplear Cloud Firestore que es un servicio de base de datos NoSQL porque nos permite una serie de ventajas como son la escalabilidad, eficiencia, flexibilidad, seguridad, entre otras. Otra de las razones de utilizar NoSQL es porque decidimos que cada actor tiene acceso a sus propios datos y aunque estén relacionados entre si, cada uno es dueño de la información que almacena en el aplicativo.
\item Notificaciones push con Firebase Cloud Messaging (FCM).
\item Hosting para alojar la aplicación web.
\end{itemize}

\subsection{Android Studio}

\subsection{FlutterFlow}
FlutterFlow~\cite{flutterflow} es una herramienta que nos permite crear aplicaciones insertando los componentes que tiene la herramienta, arrastrando y soltando. Es posible añadir lógica durante la creación de las páginas.

\subsection{LaTeX}
Entre las distintas herramientas de \LaTeX{} que existen hemos decidido utilizar TeXmaker~\cite{wiki:texmaker, texmaker} que es un editor de \LaTeX{} gratuito y multiplataforma. Además, incluye soporte unicode, corrección ortográfica, autocompletado y un visor de PDF integrado.

\subsection{Documentación}
Para la documentación del trabajo de fin de grado hemos empleado \LaTeX{}~\cite{wiki:latex} que es un sistema que permite componer textos con una alta calidad tipográfica. En el caso de la documentación del repositorio hemos empleado Markdown~\cite{wiki:markdown} que es un lenguaje de marcado ligero.

\section{Integración y entregas continuas (CI/CD)}
\subsection{Github}
Github~\cite{github} para realizar CI/CD utilizaremos los actions que tiene esta herramienta. A cada push que hagamos contra la rama main esta activará los actions creados y desplegará la aplicación tanto generando la apk de Android como la desplegará en el hosting de Firebase

\section{Herramientas contrucción del proyecto}
Nuestro proyecto está creado Flutter~\cite{flutter} que nos permite construir aplicaciones nativas multiplataforma a partir de un único código base. Flutter dispone de comandos que permite construir el proyecto y generar distintas aplicaciones como en nuestro caso son la web y Android.

\subsection{Librerías}

\subsubsection{Flutter}
Flutter dispone de un repositorio oficial para aplicaciones Dart y Flutter  \url{https://pub.dev} que contiene una infinidad de paquetes instalar en nuestro proyecto. 
Algunos paquetes que cabe destacar son:

\begin{itemize}
	\item Para securizar los datos de los usuarios que emplean la aplicación \textbf{encrypt}.
	\item Para la gestión de las citas \textbf{Syncfunsion Flutter Calendar}.
	\item Para el envío de mails \textbf{mailer}.
	\item Para almacenar nuestras variables de entorno del proyecto \textbf{dotenv}.
\end{itemize}

\subsubsection{Firebase}
Firebase~\cite{firebase} que es una plataforma de Google para desarrollar aplicaciones web y móviles que se encuentra alojada en la nube. Algunas funcionalidades para el funcionamiento de nuestra aplicación que utilizamos de esta plataforma son la autenticación, el hosting para alojar la aplicación, firestore database como base de datos, messaging para las notificaciones push.

Algunos paquetes que necesitamos instalar para poder trabajar con flutter son:
\begin{itemize}
	\item \textbf{Firebase Core}.
	\item \textbf{Firebase Auth}.
	\item \textbf{Firebase Storage}.
	\item \textbf{Cloud Firestore}.
	\item \textbf{Firebase messaging}.
\end{itemize}





\capitulo{5}{Aspectos relevantes del desarrollo del proyecto}

Este apartado pretende recoger los aspectos más relevantes del desarrollo del proyecto, pasando por la creación del proyecto, que decisiones se tomaron y la resolución de 
problemas encontrados y como se solucionaron.

\section{Inicio del proyecto}
El proyecto surgió a partir de la necesidad de ayudar a personas diagnosticadas con EM (Esclerósis Múltiple) y poder hacer su día a día más fácil a través de consejos, atención, ayuda.

Para poder hacer posible esta solución, se crea una aplicación inicialmente para el móvil, donde se incluirán secciones que pueden ayudar a los usuarios que hagan uso de ellas. Tras la asignación del proyecto se inicia un estudio de las posibilidades tecnológicas disponibles para resolver esta necesidad.

Desde el proyecto realizaremos llamadas a una API del cliente que nos proveerá de los datos necesarios para crear una aplicación amigable para los usuarios.

Tras los inconvenientes que surgieron durante el desarrollo de la idea inicial, se decidió enfocar el proyecto a pacientes diagnosticados con ELA (Esclerosis Latera Amiotrófica), para ello se definió una nueva estructura de datos, empleando documentos que se guardarían en la nube. Esta nueva estructura de datos, se centra en un principio en 3 actores que son los pacientes, los cuidadores principales de los pacientes y gestores/as de casos que gestionarán las citas que necesitan los pacientes para ayudarles a convivir con la enfermedad.

\section{Metodologías}
Desde el comienzo se realiza una arquitectura limpia para reutilizar los componentes y que sean independientes sin que haya ningún impacto si se realizan cambios en alguno de ellos.
Para la gestión del proyecto, se utiliza el panel que nos proporciona GitHub, usando la metodología ágil Scrum. El equipo al ser un proyecto creado para la realización del trabajo de fin de carrera, está formado por una persona. 
Por otro lado, el proyecto inicial se alimentaba de una API externa, realizada por un compañero de la universidad, en la que se mantienen reuniones para recabar requisitos técnicos para desarrollar la aplicación. Ante la imposibilidad de continuar con la aplicación, por inconvenientes técnicos, se considera empezar un proyecto nuevo donde seamos nosotros los que controlemos la gestión de los datos de los actores que intervienen en una primera versión del aplicativo.

Para el desarrollo de la metodología ágil se siguieron los siguientes puntos:
\begin{itemize}
\item crear iteraciones incrementales (sprints) y revisiones
\item una vez terminado el sprint se añade a la rama main que será nuestro producto final.
\item los sprints por regla general los hemos hecho de duración semanal.,
\item para cada sprint se le asocian las tareas más importantes.
\end{itemize}

\section{Formación}
Para abordar la realización y creación del proyecto se necesitaban adquirir una serie de conocimientos que inicialmente eran inexistentes. Entre los conocimientos adquiridos se destacan:
\begin{itemize}
\item Formación sobre Flutter~\cite{flutter,formacionFlutter}.
\item Formación sobre Dart~\cite{dart}.
\item Formación sobre Firebase~\cite{firebase}. Esta plataforma es la que nos permitira la gestión de los usuarios y almacenar los datos a través de documentos, además, también la gestión de notificaciones push para la creación o cancelación de citas y también para poder alojar la aplicación web.
\end{itemize}

Además, se profundiza en obtener información a cerca de que es la esclerosis múltiple y que tipos existen. Posteriormente, la aplicación se centra en recabar información de la esclerosis lateral amiotrófica.

\section{Desarrollo}
Durante el desarrollo de la aplicación nos aparecieron diversos retos de la versión incial:
\begin{itemize}
\item Conexión entre los diferentes componentes.
\item Internacionalización de la aplicación.
\item Contactar con el cliente ante desconexión de la dirección de la API facilitada inicialmente.
\item Contactar con el cliente para resolución de dudas sobre la API.
\end{itemize}

Durante el desarrollo de la aplicación de la versión final del proyecto nos aparecieron diversos retos:
\begin{itemize}
\item Conexión entre los diferentes componentes.
\item Contactar con el cliente para resolver dudas para la primera versión.
\item Acordar con el cliente que llevaría esta primera versión.
\item Trabajar con la plataforma Firebase~\cite{firebase} para la gestión de los datos, notificaciones push.
\end{itemize}

\section{Publicación}
Una vez hubo una versión lista para desplegar en producción, se creó una cuenta en Google Play y se desplegó.
Durante la creación de la cuenta de desarrollador de Google Play, nos encontramos con el reto de que no podemos verificar nuestra identidad aunque seguimos los pasos que indican durante el proceso. 

Para próximas versiones, se tratará de crear un despliegue para otros dispositivos, como es iOS.




\capitulo{6}{Trabajos relacionados}

La idea principal realizar una aplicación frontend haciendo uso del backend generado en el TFG \textbf{GII\_O\_MA\_22.05 Backend en microservicios para aplicación móvil}. A partir del uso del backend ya implementado se pretende crear una interfaz gráfica multiplataforma para usuarios afectados por la \textbf{Esclerosis Múltiple (EM)} y su posible extensión/adecuación para su uso en pacientes con la \textbf{Esclerosis Lateral Amiotrófica (ELA)}.

Ante los problemas técnicos observados durante el desarrollo de esta primera fase y bajo el consentimiento de \nomtutor, nos ponemos en contacto con la \textbf{Asociación de ELA de Castilla y León} y nos ponemos manos a la obra para crear una aplicación que inicialmente contará con una parte de administrador, que gestionará los usuarios, los hospitales, los tratamientos y roles. Así mismo, también crearemos una aplicación para los usuarios y que tendrá la funcionalidad de la gestión de citas. 

Algunas aplicaciones que están relacionadas son:
\begin{itemize}
\item \textbf{ME -- Multiple Esclerosis}. Para realizar un seguimiento y controlar sus síntomas.
\item \textbf{Cleo}. Centrada en el diálogo directo y con servicio de enfermería.
\item \textbf{Control EM}. También centrada en el control de la enfermedad.
\end{itemize}

\capitulo{7}{Conclusiones y Líneas de trabajo futuras}

Todo proyecto debe incluir las conclusiones que se derivan de su desarrollo. Éstas pueden ser de diferente índole, dependiendo de la tipología del proyecto, pero normalmente van a estar presentes un conjunto de conclusiones relacionadas con los resultados del proyecto y un conjunto de conclusiones técnicas. 

\section{Conclusiones}
Las conclusiones que sacamos durante el trayecto del desarrollo de las aplicaciones para el trabajo final de grado, son que una aplicación no siempre que se tiene una idea inicial se consigue llegar a termino y finalizarla exitosamente. Por fortuna, estamos en un mundo en el que tenemos que estar abierto a cambios y a enfrentarnos a retos para poder dar solución a los problemas de los usuarios finales.

A continuación, vamos a enumerar conclusiones obtenidas durante la realización del trabajo:
\begin{itemize}
\item Ante los inconvenientes durante el transcurso del proyecto, podemos afirmar que estamos contentos con el resultado obtenido y con el rumbo con el que hemos comenzado. Para empezar esta primera versión se centra en la gestión de las citas que tiene el usuario programadas en el tiempo.
\item El haber utilizado el framework Flutter para desarrollar la aplicación nos permite crear una aplicación multiplataforma siendo de código abierto, se integra bien con otras tecnologías, como en nuestro caso es Firebase y una vez adquiriendo la experiencia suficiente facilita el desarrollo.
\item Se emplean los conocimientos utilizados durante el grado para el desarrollo de la aplicación.
\end{itemize}

\section{Líneas de trabajo futuras}
Como en todas las aplicaciones se parte de una versión inicial y esta aplicación puede seguir creciendo para ir incluyendo nuevas funcionalidades y desarrollos que facilitarán el día a día de los usuarios.
\begin{itemize}
\item Crear opción para ver el material ortoprotésico
\item Crear opción para ver el listado de las unidades de ELA disponibles
\item Crear opción para ver el listado de las los ensayos clínicos
\item Crear opción para ver el listado de las artículos de investigación
\item Crear opción para mantener una conversación con la gestora de casos para agendar una cita.
\end{itemize}



\bibliographystyle{plain}
\bibliography{bibliografia}

\end{document}
